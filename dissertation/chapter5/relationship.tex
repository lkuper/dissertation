\section{Discussion: reasoning about per-query consistency choices}\label{s:distributed-per-query-consistency}

In this chapter, we have seen a way to extend CvRDTs to support
LVar-style threshold queries.  Seen from another angle, this chapter
shows how to ``port'' the notion of threshold reads from a
shared-memory setting (that of LVars) to a distributed-memory one
(that of CvRDTs).  However, I do not want to suggest that
deterministic threshold queries should replace traditional CvRDT
queries.  Instead, traditional queries and threshold queries can
coexist.  Moreover, extending CvRDTs with threshold queries allows
them to more accurately model systems in which consistency properties
are defined and enforced at the granularity of individual queries.

As mentioned at the beginning of this chapter, database services such
as Amazon's SimpleDB~\cite{simpledb-vogels-article} allow for both
eventually consistent and strongly consistent reads, chosen at a
per-query granularity.\footnote{Terry~\etal's Pileus key-value
  store~\cite{pileus} takes the idea of combining different levels of
  consistency in a single application even further: instead of
  requiring the application developer to choose the consistency level
  of a particular query at development time, the system allows the
  developer to specify a service-level agreement that may be satisfied
  in different ways at runtime.  This allows the application to, for
  instance, dynamically adapt to changing network conditions.}
Choosing consistency at the query level, and giving different
consistency properties to different queries within a single
application, is not a new idea.  Rather, the new contribution we make
by adding threshold queries to CvRDTs is to establish lattice-based
data structures as a unifying formal foundation for both eventually
consistent and strongly consistent queries.  Adding support for
threshold reads to CvRDTs allows us to take advantage of the machinery
that CvRDTs already give us for reasoning about eventually consistent
objects, and use it to reason about systems that allow consistency
choices to be made at per-query granularity, as real systems do.

