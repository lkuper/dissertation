\begin{proof}
  Suppose $\conf \ctxstepsto \conf_a$ and $\conf \ctxstepsto \conf_b$.
  We have to show that either there exist $\conf_c, i, j, \pi$ such
  that $\conf_a \ctxstepsto^i \conf_c$ and $\pi(\conf_b) \ctxstepsto^j
  \conf_c$ and $i \leq 1$ and $j \leq 1$, or that $\conf_a \ctxstepsto
  \error$ or $\conf_b \ctxstepsto \error$.

  By inspection of the operational semantics, it must be the case that
  $\conf$ steps to $\conf_a$ by the {\sc E-Eval-Ctxt} rule.  Let
  $\conf = \config{S}{\evalctxt{E_a}{e_{a_1}}}$ and let $\conf_a =
  \config{S_a}{\evalctxt{E_a}{e_{a_2}}}$.

  Likewise, it must be the case that $\conf$ steps to $\conf_b$ by the
  {\sc E-Eval-Ctxt} rule.  Let $\conf =
  \config{S}{\evalctxt{E_b}{e_{b_1}}}$ and let $\conf_b =
  \config{S_b}{\evalctxt{E_b}{e_{b_2}}}$.

  Note that $\conf = \config{S}{\evalctxt{E_a}{e_{a_1}}} =
  \config{S}{\evalctxt{E_b}{e_{b_1}}}$, and so
  $\evalctxt{E_a}{e_{a_1}} = \evalctxt{E_b}{e_{b_1}}$, but $E_a$ and
  $E_b$ may differ and $e_{a_1}$ and $e_{b_1}$ may differ.

  Since $\config{S}{\evalctxt{E_a}{e_{a_1}}} \ctxstepsto
  \config{S_a}{\evalctxt{E_a}{e_{a_2}}}$ and
  $\config{S}{\evalctxt{E_b}{e_{b_1}}} \ctxstepsto
  \config{S_b}{\evalctxt{E_b}{e_{b_2}}}$ and $\evalctxt{E_a}{e_{a_1}}
  = \evalctxt{E_b}{e_{b_1}}$, we have from Lemma~\ref{lem:locality}
  (Locality) that there exist evaluation contexts $E'_a$ and $E'_b$
  such that:

  \begin{itemize}
  \item $\evalctxt{E'_a}{e_{a_1}} = \evalctxt{E_b}{e_{b_2}}$, and
  \item $\evalctxt{E'_b}{e_{b_1}} = \evalctxt{E_a}{e_{a_2}}$, and
  \item $\evalctxt{E'_a}{e_{a_2}} =
  \evalctxt{E'_b}{e_{b_2}}$.
  \end{itemize}

  In some of the cases that follow, we will show that $\conf_a
  \ctxstepsto \error$ or that $\conf_b \ctxstepsto \error$.  In
  others, we will choose $\conf_c = \error$.  In most cases, however,
  our approach will be to show that there exist $S', i, j, \pi$ such
  that:
  \begin{itemize}
  \item $\config{S_a}{\evalctxt{E_a}{e_{a_2}}} \ctxstepsto^i
    \config{S'}{\evalctxt{E'_a}{e_{a_2}}}$, and
  \item $\pi(\config{S_b}{\evalctxt{E_b}{e_{b_2}}}) \ctxstepsto^j
    \config{S'}{\evalctxt{E'_a}{e_{a_2}}}$.
  \end{itemize}
  Since $\evalctxt{E'_a}{e_{a_1}} = \evalctxt{E_b}{e_{b_2}}$,
  $\evalctxt{E'_b}{e_{b_1}} = \evalctxt{E_a}{e_{a_2}}$, and
  $\evalctxt{E'_a}{e_{a_2}} = \evalctxt{E'_b}{e_{b_2}}$, it suffices
  to show that:
  \begin{itemize}
  \item $\config{S_a}{\evalctxt{E'_b}{e_{b_1}}} \ctxstepsto^i
    \config{S'}{\evalctxt{E'_b}{e_{b_2}}}$, and
  \item $\pi(\config{S_b}{\evalctxt{E'_a}{e_{a_1}}}) \ctxstepsto^j
    \config{S'}{\evalctxt{E'_a}{e_{a_2}}}$.
  \end{itemize}

  From the premise of {\sc E-Eval-Ctxt}, we have that
  $\config{S}{e_{a_1}} \parstepsto \config{S_a}{e_{a_2}}$ and
  $\config{S}{e_{b_1}} \parstepsto \config{S_b}{e_{b_2}}$.  We proceed
  by case analysis on the rule by which $\config{S}{e_{a_1}}$ steps to
  $\config{S_a}{e_{a_2}}$.

  \begin{enumerate}
  \item Case {\sc E-Beta}: We have $S_a = S$.

    We proceed by case analysis on the rule by which
    $\config{S}{e_{b_1}}$ steps to $\config{S_b}{e_{b_2}}$:
    \begin{enumerate}
    \item \label{slqc-beta-beta}Case {\sc E-Beta}: We have $S_b =
      S$.

      Choose $S' = S = S_a = S_b$, $i = 1$, $j = 1$, and $\pi = \id$.

      We have to show that:
      \begin{itemize}
      \item $\config{S}{\evalctxt{E'_b}{e_{b_1}}} \ctxstepsto
        \config{S_a}{\evalctxt{E'_b}{e_{b_2}}}$, and
      \item $\config{S}{\evalctxt{E'_a}{e_{a_1}}} \ctxstepsto
        \config{S_b}{\evalctxt{E'_a}{e_{a_2}}}$, 
      \end{itemize}

      both of which follow immediately from $\config{S}{e_{a_1}}
      \parstepsto \config{S_a}{e_{a_2}}$ and $\config{S}{e_{b_1}}
      \parstepsto \config{S_b}{e_{b_2}}$ and {\sc E-Eval-Ctxt}.

    \item \label{slqc-beta-new}Case {\sc E-New}: We have $S_b =
      \extS{S}{l}{\bot}{\frozenfalse}$.

      Choose $S' = S_b$, $i = 1$, $j = 1$, and $\pi = \id$.

      We have to show that:
      \begin{itemize}
      \item $\config{S}{\evalctxt{E'_b}{e_{b_1}}} \ctxstepsto
        \config{S_b}{\evalctxt{E'_b}{e_{b_2}}}$, and
      \item
        $\config{S_b}{\evalctxt{E'_a}{e_{a_1}}} \ctxstepsto
        \config{S_b}{\evalctxt{E'_a}{e_{a_2}}}$.
      \end{itemize}

      The first of these follows immediately from $\config{S}{e_{b_1}}
      \parstepsto \config{S_b}{e_{b_2}}$ and {\sc E-Eval-Ctxt}.  For
      the second, consider that $S_b = \extS{S}{l}{\bot}{\frozenfalse} =
      \lubstore{S}{\store{\storebinding{l}{\bot}{\frozenfalse}}}$.  Furthermore,
      since no locations are allocated during the transition
      $\config{S}{e_{a_1}} \parstepsto \config{S_a}{e_{a_2}}$, we know
      that $\store{\storebinding{l}{\bot}{\frozenfalse}}$ is non-conflicting with
      it, and we know that
      $\lubstore{S_a}{\store{\storebinding{l}{\bot}{\frozenfalse}}} \neq \topS$
      since $S_a$ is just $S$ and
      $\lubstore{S}{\store{\storebinding{l}{\bot}{\frozenfalse}}}$ cannot be
      $\topS$.  Therefore, by Lemma~\ref{lem:independence}
      (Independence), we have that
      $\config{\lubstore{S}{\store{\storebinding{l}{\bot}{\frozenfalse}}}}{e_{a_1}}
      \parstepsto
      \config{\lubstore{S_a}{\store{\storebinding{l}{\bot}{\frozenfalse}}}}{e_{a_2}}$.
      Hence $\config{S_b}{e_{a_1}} \parstepsto \config{S_b}{e_{a_2}}$.
      By {\sc E-Eval-Ctxt}, it follows that
      $\config{S_b}{\evalctxt{E'_a}{e_{a_1}}} \ctxstepsto
      \config{S_b}{\evalctxt{E'_a}{e_{a_2}}}$, as we were required to
      show.

    \item \label{slqc-beta-put}Case {\sc E-Put}: 

      \TODO{Finish updating this case to use $\putiexp{}$.}

      We have $S_b = \extSRaw{S}{l}{u_{p_i}(p_1)}$.

      Choose $S' = S_b$, $i = 1$, $j = 1$, and $\pi = \id$.

      We have to show that:
      \begin{itemize}
      \item $\config{S}{\evalctxt{E'_b}{e_{b_1}}} \ctxstepsto
        \config{S_b}{\evalctxt{E'_b}{e_{b_2}}}$, and
      \item
        $\config{S_b}{\evalctxt{E'_a}{e_{a_1}}} \ctxstepsto
        \config{S_b}{\evalctxt{E'_a}{e_{a_2}}}$.
      \end{itemize}

      The first of these follows immediately from $\config{S}{e_{b_1}}
      \parstepsto \config{S_b}{e_{b_2}}$ and {\sc E-Eval-Ctxt}.  For
      the second, consider that $S_b =
      \extSRaw{S}{l}{u_{p_i}(p_1)} =
      \lubstore{S}{\store{\storebindingRaw{l}{u_{p_i}(p_1)}}}$.
      Furthermore, since no locations are allocated during the
      transition $\config{S}{e_{a_1}} \parstepsto
      \config{S_a}{e_{a_2}}$, we know that
      $\store{\storebindingRaw{l}{u_{p_i}(p_1)}}$ is
      non-conflicting with it, and we know that
      $\lubstore{S_a}{\store{\storebindingRaw{l}{u_{p_i}(p_1)}}}
      \neq \topS$ since $S_a$ is just $S$ and
      $\lubstore{S}{\store{\storebindingRaw{l}{u_{p_i}(p_1)}}}$
      cannot be $\topS$, since we know from the premise of {\sc E-Put}
      that $u_{p_i}(p_1) \neq \top$.  Therefore, by
      Lemma~\ref{lem:independence} (Independence), we have that
      $\config{\lubstore{S}{\store{\storebindingRaw{l}{u_{p_i}(p_1)}}}}{e_{a_1}}
      \parstepsto
      \config{\lubstore{S_a}{\store{\storebindingRaw{l}{u_{p_i}(p_1)}}}}{e_{a_2}}$.
      Hence $\config{S_b}{e_{a_1}} \parstepsto \config{S_b}{e_{a_2}}$.
      By {\sc E-Eval-Ctxt}, it follows that
      $\config{S_b}{\evalctxt{E'_a}{e_{a_1}}} \ctxstepsto
      \config{S_b}{\evalctxt{E'_a}{e_{a_2}}}$, as we were required to
      show.

    \item \label{slqc-beta-put-err}Case {\sc E-Put-Err}:

      Here $\config{S_b}{e_{b_2}} = \error$, and so we choose $\conf_c
      = \error$, $i = 1$, $j = 0$, and $\pi = \id$.  We have to show that:
      \begin{itemize}
      \item $\config{S}{\evalctxt{E'_b}{e_{b_1}}} \ctxstepsto \error$,
        and
      \item
        $\config{S_b}{\evalctxt{E'_a}{e_{a_1}}} = \error$.
      \end{itemize}

      The second of these is immediately true because since
      $\config{S_b}{e_{b_2}} = \error$, $S_b = \topS$, and so
      $\config{S_b}{\evalctxt{E'_a}{e_{a_1}}}$ is equal to $\error$ as
      well.  For the first, observe that $\config{S}{e_{b_1}}
      \parstepsto \config{S_b}{e_{b_2}}$, hence by {\sc E-Eval-Ctxt},
      $\config{S}{\evalctxt{E'_b}{e_{b_1}}} \ctxstepsto
      \config{S_b}{\evalctxt{E'_b}{e_{b_2}}}$.  But $S_b = \topS$, so
      $\config{S_b}{\evalctxt{E'_b}{e_{b_2}}}$ is equal to $\error$,
      and so $\config{S}{\evalctxt{E'_b}{e_{b_1}}} \ctxstepsto
      \error$, as required.

    \item \label{slqc-beta-get}Case {\sc E-Get}: Similar to
      case~\ref{slqc-beta-beta}, since $S_b = S$.
    \item \label{slqc-beta-freeze-init}Case {\sc E-Freeze-Init}:
      Similar to case~\ref{slqc-beta-beta}, since $S_b = S$.
    \item \label{slqc-beta-spawn-handler}Case {\sc E-Spawn-Handler}:
      Similar to case~\ref{slqc-beta-beta}, since $S_b = S$.
    \item \label{slqc-beta-freeze-final}Case {\sc E-Freeze-Final}: \TODO{}
    \item \label{slqc-beta-freeze-simple}Case {\sc E-Freeze-Simple}: \TODO{}


    \end{enumerate}
  \item Case {\sc E-New}: We have $S_a = \extS{S}{l}{\bot}{\frozenfalse}$.

    We proceed by case analysis on the rule by which
    $\config{S}{e_{b_1}}$ steps to $\config{S_b}{e_{b_2}}$:
    \begin{enumerate}
    \item \label{slqc-new-beta}Case {\sc E-Beta}: By symmetry with case~\ref{slqc-beta-new}.
    \item \label{slqc-new-new}Case {\sc E-New}: We have $S_b = \extS{S}{l'}{\bot}{\frozenfalse}$.

      Now consider whether $l = l'$:
      \begin{itemize}
        \item If $l \neq l'$:

          Choose $S' = \extS{\extS{S}{l'}{\bot}{\frozenfalse}}{l}{\bot}{\frozenfalse}$, $i =
          1$, $j = 1$, and $\pi = \id$.

          We have to show that:
          \begin{itemize}
          \item
            $\config{S_a}{\evalctxt{E'_b}{e_{b_1}}}
            \ctxstepsto
            \config{\extS{\extS{S}{l'}{\bot}{\frozenfalse}}{l}{\bot}{\frozenfalse}}{\evalctxt{E'_b}{e_{b_2}}}$,
            and
          \item
            $\config{S_b}{\evalctxt{E'_a}{e_{a_1}}}
            \ctxstepsto
            \config{\extS{\extS{S}{l'}{\bot}{\frozenfalse}}{l}{\bot}{\frozenfalse}}{\evalctxt{E'_a}{e_{a_2}}}$.
          \end{itemize}

          For the first of these, consider that $S_a =
          \extS{S}{l}{\bot}{\frozenfalse} =
          \lubstore{S}{\store{\storebinding{l}{\bot}{\frozenfalse}}}$, and that
          $\extS{\extS{S}{l'}{\bot}{\frozenfalse}}{l}{\bot}{\frozenfalse} =
          \lubstore{\extS{S}{l'}{\bot}{\frozenfalse}}{\store{\storebinding{l}{\bot}{\frozenfalse}}}$.
          Furthermore, since the only location allocated during the
          transition $\config{S}{e_{b_1}} \parstepsto
          \config{S_b}{e_{b_2}}$ is $l'$, we know that
          $\store{\storebinding{l}{\bot}{\frozenfalse}}$ is non-conflicting with
          it (since $l \neq l'$ in this case).  We also know that
          $\lubstore{\extS{S}{l'}{\bot}{\frozenfalse}}{\store{\storebinding{l}{\bot}{\frozenfalse}}}
          \neq \topS$, since $S \neq \topS$ and new bindings of
          $\storebinding{l}{\bot}{\frozenfalse}$ and $\storebindingRaw{l'}{\bot}$
          cannot cause it to become $\topS$.  Therefore, by
          Lemma~\ref{lem:independence} (Independence), we have
          that
          $\config{\lubstore{S}{\store{\storebinding{l}{\bot}{\frozenfalse}}}}{e_{b_1}}
          \parstepsto
          \config{\lubstore{S_b}{\store{\storebinding{l}{\bot}{\frozenfalse}}}}{e_{b_2}}$.
          Hence $\config{\extS{S}{l}{\bot}{\frozenfalse}}{e_{b_1}} \parstepsto
          \config{\extS{S_b}{l}{\bot}{\frozenfalse}}{e_{b_2}}$.  By {\sc
            E-Eval-Ctxt} it follows that
          $\config{\extS{S}{l}{\bot}{\frozenfalse}}{\evalctxt{E'_b}{e_{b_1}}}
          \parstepsto
          \config{\extS{S_b}{l}{\bot}{\frozenfalse}}{\evalctxt{E'_b}{e_{b_2}}}$,
          which, since $S_b = \extS{S}{l'}{\bot}{\frozenfalse}$, is what we were
          required to show.  The argument for the second is
          symmetrical.

        \item If $l = l'$:

          In this case, observe that we do \emph{not} want the
          expression in the final configuration to be
          $\evalctxt{E'_a}{e_{a_2}}$ (nor its equivalent,
          $\evalctxt{E'_b}{e_{b_2}}$).  The reason for this is that
          $\evalctxt{E'_a}{e_{a_2}}$ contains both occurrences of $l$.
          Rather, we want both configurations to step to a
          configuration in which exactly one occurrence of $l$ has
          been renamed to a fresh location $l''$.

          Let $l''$ be a location such that $l'' \notin S$ and $l''
          \neq l$ (and hence $l'' \neq l'$, as well).  Then choose $S'
          = \extS{\extS{S}{l''}{\bot}{\frozenfalse}}{l}{\bot}{\frozenfalse}$, $i = 1$, $j =
          1$, and $\pi = \setof{(l, l'')}$.

          Either
          $\config{\extS{\extS{S}{l''}{\bot}{\frozenfalse}}{l}{\bot}{\frozenfalse}}{\evalctxt{E'_a}{\pi(e_{a_2})}}$
          or
          $\config{\extS{\extS{S}{l''}{\bot}{\frozenfalse}}{l}{\bot}{\frozenfalse}}{\evalctxt{E'_b}{\pi(e_{b_2})}}$
          would work as a final configuration; we choose
          $\config{\extS{\extS{S}{l''}{\bot}{\frozenfalse}}{l}{\bot}{\frozenfalse}}{\evalctxt{E'_b}{\pi(e_{b_2})}}$.

          We have to show that:
          \begin{itemize}
          \item
            $\config{S_a}{\evalctxt{E'_b}{e_{b_1}}} \ctxstepsto
            \config{\extS{\extS{S}{l''}{\bot}{\frozenfalse}}{l}{\bot}{\frozenfalse}}{\evalctxt{E'_b}{\pi(e_{b_2})}}$,
            and
          \item
            $\pi(\config{S_b}{\evalctxt{E'_a}{e_{a_1}}}) \ctxstepsto
            \config{\extS{\extS{S}{l''}{\bot}{\frozenfalse}}{l}{\bot}{\frozenfalse}}{\evalctxt{E'_b}{\pi(e_{b_2})}}$.
          \end{itemize}

          For the first of these, since $\config{S}{e_{b_1}}
          \parstepsto \config{S_b}{e_{b_2}}$, we have by
          Lemma~\ref{lem:permutability} (Permutability) that
          $\pi(\config{S}{e_{b_1}}) \parstepsto
          \pi(\config{S_b}{e_{b_2}})$.  Since $\pi = \setof{(l,
            l'')}$, but $l \notin S$ (from the side condition on {\sc
            E-New}), we have that $\pi(\config{S}{e_{b_1}}) =
          \config{S}{e_{b_1}}$. Since $\config{S_b}{e_{b_2}} =
          \config{\extS{S}{l'}{\bot}{\frozenfalse}}{l'}$, and $l = l'$, we have
          that $\pi(\config{S_b}{e_{b_2}}) =
          \config{\extS{S}{l''}{\bot}{\frozenfalse}}{\pi(e_{b_2})}$.  Hence
          $\config{S}{e_{b_1}} \parstepsto
          \config{\extS{S}{l''}{\bot}{\frozenfalse}}{\pi(e_{b_2})}$.

          Since the only location allocated during the transition
          $\config{S}{e_{b_1}} \parstepsto
          \config{\extS{S}{l''}{\bot}{\frozenfalse}}{\pi(e_{b_2})}$ is $l''$, we
          know that $\store{\storebinding{l}{\bot}{\frozenfalse}}$ is
          non-conflicting with it.  We also know that
          $\lubstore{\extS{S}{l''}{\bot}{\frozenfalse}}{\store{\storebinding{l}{\bot}{\frozenfalse}}}
          \neq \topS$, since $S \neq \topS$ and new bindings of
          $\storebindingRaw{l''}{\bot}$ and
          $\storebinding{l}{\bot}{\frozenfalse}$ cannot cause it to become
          $\topS$.  Therefore, by Lemma~\ref{lem:independence}
          (Independence), we have that
          $\config{\lubstore{S}{\store{\storebinding{l}{\bot}{\frozenfalse}}}}{e_{b_1}}
          \parstepsto
          \config{\lubstore{\extS{S}{l''}{\bot}{\frozenfalse}}{\store{\storebinding{l}{\bot}{\frozenfalse}}}}{\pi(e_{b_2})}$.
          Hence $\config{\extS{S}{l}{\bot}{\frozenfalse}}{e_{b_1}} \parstepsto
          \config{\extS{\extS{S}{l''}{\bot}{\frozenfalse}}{l}{\bot}{\frozenfalse}}{\pi(e_{b_2})}$.
          By {\sc E-Eval-Ctxt} it follows that
          $\config{\extS{S}{l}{\bot}{\frozenfalse}}{\evalctxt{E'_b}{e_{b_1}}}
          \parstepsto
          \config{\extS{\extS{S}{l''}{\bot}{\frozenfalse}}{l}{\bot}{\frozenfalse}}{\evalctxt{E'_b}{\pi(e_{b_2})}}$,
          which, since $\extS{S}{l}{\bot}{\frozenfalse} = S_a$, is what we were
          required to show.

          For the second, observe that since $S_b =
          \extS{S}{l}{\bot}{\frozenfalse}$, we have that $\pi(S_b) =
          \extS{S}{l''}{\bot}{\frozenfalse}$.  Also, since $l$ does not occur in
          $e_{a_1}$, we have that $\pi(\evalctxt{E'_a}{e_{a_1}}) =
          \evalctxt{(\pi(E'_a))}{e_{a_1}}$.  Hence we have to show that

          $\config{\extS{S}{l''}{\bot}{\frozenfalse}}{\evalctxt{(\pi(E'_a))}{e_{a_1}}}
          \ctxstepsto
          \config{\extS{\extS{S}{l''}{\bot}{\frozenfalse}}{l}{\bot}{\frozenfalse}}{\evalctxt{E'_b}{\pi(e_{b_2})}}$.

          Since the only location allocated during the transition
          $\config{S}{e_{a_1}} \parstepsto \config{S_a}{e_{a_2}}$ is
          $l$, we know that $\store{\storebindingRaw{l''}{\bot}}$ is
          non-conflicting with it.  We also know that
          $\lubstore{S_a}{\store{\storebindingRaw{l''}{\bot}}} \neq
          \topS$, since $S_a = \extS{S}{l}{\bot}{\frozenfalse}$ and $S \neq
          \topS$ and new bindings of $\storebindingRaw{l''}{\bot}$ and
          $\storebinding{l}{\bot}{\frozenfalse}$ cannot cause it to become
          $\topS$.  Therefore, by Lemma~\ref{lem:independence}
          (Independence), we have that
          $\config{\lubstore{S}{\store{\storebindingRaw{l''}{\bot}}}}{e_{a_1}}
          \parstepsto
          \config{\lubstore{S_a}{\store{\storebindingRaw{l''}{\bot}}}}{e_{a_2}}$.
          Hence $\config{\extS{S}{l''}{\bot}{\frozenfalse}}{e_{a_1}} \parstepsto
          \config{\extS{\extS{S}{l''}{\bot}{\frozenfalse}}{l}{\bot}{\frozenfalse}}{e_{a_2}}$.
          By {\sc E-Eval-Ctxt} it follows that
          $\config{\extS{S}{l''}{\bot}{\frozenfalse}}{\evalctxt{(\pi(E'_a))}{e_{a_1}}}
          \ctxstepsto
          \config{\extS{\extS{S}{l''}{\bot}{\frozenfalse}}{l}{\bot}{\frozenfalse}}{\evalctxt{(\pi(E'_a))}{e_{a_2}}}$,
          which completes the case since
          $\evalctxt{E'_b}{\pi(e_{b_2})} =
          \evalctxt{(\pi(E'_a))}{e_{a_2}}$.

          \lk{This is really sketchy -- I should really explain why
            $\evalctxt{E'_b}{\pi(e_{b_2})} =
            \evalctxt{(\pi(E'_a))}{e_{a_2}}$.}

      \end{itemize}

    \item \label{slqc-new-put}Case {\sc E-Put}:

      \TODO{Finish updating this case to use $\putiexp{}$.}

      We have $S_b = \extSRaw{S}{l'}{u_{p_i}(p_1)}$.

      We have to show that:
      \begin{itemize}
      \item $\config{S_a}{\evalctxt{E'_b}{e_{b_1}}} \ctxstepsto
        \config{\extS{S_b}{l}{\bot}{\frozenfalse}}{\evalctxt{E'_b}{e_{b_2}}}$,
        and
      \item
        $\config{S_b}{\evalctxt{E'_a}{e_{a_1}}} \ctxstepsto
        \config{\extS{S_b}{l}{\bot}{\frozenfalse}}{\evalctxt{E'_a}{e_{a_2}}}$.
      \end{itemize}

      For the first of these, consider that $S_a =
      \extS{S}{l}{\bot}{\frozenfalse} =
      \lubstore{S}{\store{\storebinding{l}{\bot}{\frozenfalse}}}$, and that
      $\extS{S_b}{l}{\bot}{\frozenfalse} =
      \lubstore{S_b}{\store{\storebinding{l}{\bot}{\frozenfalse}}}$.
      Furthermore, since no locations are allocated during the
      transition $\config{S}{e_{b_1}} \parstepsto
      \config{S_b}{e_{b_2}}$, we know that
      $\store{\storebinding{l}{\bot}{\frozenfalse}}$ is non-conflicting with it.
      We also know that
      $\lubstore{S_b}{\store{\storebinding{l}{\bot}{\frozenfalse}}} \neq \topS$,
      since $S_b = \extSRaw{S}{l'}{u_{p_i}(p_1)}$ and we know
      from the premise of {\sc E-Put} that $u_{p_i}(p_1) \neq
      \top$.  Therefore, by Lemma~\ref{lem:independence}
      (Independence), we have that
      $\config{\lubstore{S}{\store{\storebinding{l}{\bot}{\frozenfalse}}}}{e_{b_1}}
      \parstepsto
      \config{\lubstore{S_b}{\store{\storebinding{l}{\bot}{\frozenfalse}}}}{e_{b_2}}$.
      Hence $\config{\extS{S}{l}{\bot}{\frozenfalse}}{e_{b_1}} \parstepsto
      \config{\extS{S_b}{l}{\bot}{\frozenfalse}}{e_{b_2}}$.  By {\sc
        E-Eval-Ctxt}, it follows that
      $\config{\extS{S}{l}{\bot}{\frozenfalse}}{\evalctxt{E'_b}{e_{b_1}}}
      \ctxstepsto
      \config{\extS{S_b}{l}{\bot}{\frozenfalse}}{\evalctxt{E'_b}{e_{b_2}}}$,
      which, since $S_a = \extS{S}{l}{\bot}{\frozenfalse}$, is what we were
      required to show.

      For the second, consider that $S_b =
      \lubstore{S}{\store{\storebindingRaw{l'}{u_{p_i}(p_1)}}}$
      and $\extS{S_b}{l}{\bot}{\frozenfalse} =
      \lubstore{\extS{S}{l}{\bot}{\frozenfalse}}{\store{\storebindingRaw{l'}{u_{p_i}(p_1)}}}
      =
      \lubstore{S_a}{\store{\storebindingRaw{l'}{u_{p_i}(p_1)}}}$.
      Furthermore, since the only location allocated during the
      transition $\config{S}{e_{a_1}} \parstepsto
      \config{S_a}{e_{a_2}}$ is $l$, we know that
      $\store{\storebindingRaw{l'}{u_{p_i}(p_1)}}$ is
      non-conflicting with it.  (We know that $l \neq l'$ because we
      have from the premise of {\sc E-Put} that $l' \in \dom{S}$, but
      we have from the side condition of {\sc E-New} that $l \notin
      \dom{S}$.)  We also know that
      $\lubstore{\extS{S}{l}{\bot}{\frozenfalse}}{\store{\storebindingRaw{l'}{u_{p_i}(p_1)}}}
      \neq \topS$, since we know from the premise of {\sc E-Put} that
      $u_{p_i}(p_1) \neq \top$.  Therefore, by
      Lemma~\ref{lem:independence} (Independence), we have that
      $\config{\lubstore{S}{\store{\storebindingRaw{l'}{u_{p_i}(p_1)}}}}{e_{a_1}}
      \parstepsto
      \config{\lubstore{S_a}{\store{\storebindingRaw{l'}{u_{p_i}(p_1)}}}}{e_{a_2}}$.
      Hence $\config{S_b}{e_{a_1}} \parstepsto
      \config{\extS{S_b}{l}{\bot}{\frozenfalse}}{e_{a_2}}$.  By {\sc
        E-Eval-Ctxt}, it follows that
      $\config{S_b}{\evalctxt{E'_a}{e_{a_1}}} \ctxstepsto
      \config{\extS{S_b}{l}{\bot}{\frozenfalse}}{\evalctxt{E'_a}{e_{a_2}}}$, as
      we were required to show.

    \item \label{slqc-new-put-err}Case {\sc E-Put-Err}:

      Here $\config{S_b}{e_{b_2}} = \error$, and so we choose $\conf_c
      = \error$, $i = 1$, $j = 0$, and $\pi = \id$.  We have to show
      that:
      \begin{itemize}
      \item $\config{S_a}{\evalctxt{E'_b}{e_{b_1}}} \ctxstepsto
        \error$, and
      \item
        $\config{S_b}{\evalctxt{E'_a}{e_{a_1}}} = \error$.
      \end{itemize}

      The second of these is immediately true because since
      $\config{S_b}{e_{b_2}} = \error$, $S_b = \topS$, and so
      $\config{S_b}{\evalctxt{E'_a}{e_{a_1}}}$ is equal to $\error$ as
      well.  For the first, observe that since $\config{S}{e_{a_1}}
      \parstepsto \config{S_a}{e_{a_2}}$, we have by
      Lemma~\ref{lem:monotonicity} (Monotonicity) that
      $\leqstore{S}{S_a}$.  Therefore, since $\config{S}{e_{b_1}}
      \parstepsto \error$, we have by
      (Error Preservation)
      \TODO{Don't use Error Preservation.}
      that $\config{S_a}{e_{b_1}} \parstepsto \error$.  Since $\error$
      is equal to $\config{\topS}{e}$ for all expressions $e$,
      $\config{S_a}{e_{b_1}} \parstepsto \config{\topS}{e}$ for all
      $e$.  Therefore, by {\sc E-Eval-Ctxt},
      $\config{S_a}{\evalctxt{E'_b}{e_{b_1}}} \ctxstepsto
      \config{\topS}{\evalctxt{E'_b}{e}}$ for all $e$.  Since
      $\config{\topS}{\evalctxt{E'_b}{e}}$ is equal to $\error$, we
      have that $\config{S_a}{\evalctxt{E'_b}{e_{b_1}}} \ctxstepsto
      \error$, as we were required to show.

    \item \label{slqc-new-get}Case {\sc E-Get}: Similar to
      case~\ref{slqc-new-beta}, since $S_b = S$.
    \item \label{slqc-new-freeze-init}Case {\sc E-Freeze-Init}:
      Similar to case~\ref{slqc-new-beta}, since $S_b = S$.
    \item \label{slqc-new-spawn-handler}Case {\sc E-Spawn-Handler}:
      Similar to case~\ref{slqc-new-beta}, since $S_b = S$.
    \item \label{slqc-new-freeze-final}Case {\sc E-Freeze-Final}: \TODO{}
    \item \label{slqc-new-freeze-simple}Case {\sc E-Freeze-Simple}: \TODO{}

    \end{enumerate}
  \item Case {\sc E-Put}:

    \TODO{Finish updating this case to use $\putiexp{}$.}

    We have $S_a = \extSRaw{S}{l}{u_{p_i}(p_1)}$.

    We proceed by case analysis on the rule by which
    $\config{S}{e_{b_1}}$ steps to $\config{S_b}{e_{b_2}}$:
    \begin{enumerate}
    \item \label{slqc-put-beta}Case {\sc E-Beta}: By symmetry with case~\ref{slqc-beta-put}.
    \item \label{slqc-put-new}Case {\sc E-New}: By symmetry with case~\ref{slqc-new-put}.
    \item \label{slqc-put-put}Case {\sc E-Put}:

      \TODO{Finish updating this case to use $\putiexp{}$.}

      We have $S_b = \extSRaw{S}{l'}{\userlub{d'_1}{d'_2}}$, where
      $d'_1 = S(l')$.

      Consider whether
      $\lubstore{S_b}{\store{\storebindingRaw{l}{u_{p_i}(p_1)}}}
      = \topS$:

      \begin{itemize}
      \item
        $\lubstore{S_b}{\store{\storebindingRaw{l}{u_{p_i}(p_1)}}}
        \neq \topS$:

        Choose $S' = \lubstore{S_a}{S_b}$, $i = 1$, $j = 1$, and $\pi =
        \id$.

        We have to show that:
        \begin{itemize}
        \item $\config{S_a}{\evalctxt{E'_b}{e_{b_1}}} \ctxstepsto
          \config{\lubstore{S_a}{S_b}}{\evalctxt{E'_b}{e_{b_2}}}$, and
        \item $\config{S_b}{\evalctxt{E'_a}{e_{a_1}}} \ctxstepsto
          \config{\lubstore{S_a}{S_b}}{\evalctxt{E'_a}{e_{a_2}}}$.
        \end{itemize}

        For the first of these, since no locations are allocated
        during the transition $\config{S}{e_{b_1}} \parstepsto
        \config{S_b}{e_{b_2}}$, we know that
        $\store{\storebindingRaw{l}{u_{p_i}(p_1)}}$ is
        non-conflicting with it, and in this subcase, we know that
        $\lubstore{S_b}{\store{\storebindingRaw{l}{u_{p_i}(p_1)}}}
        \neq \topS$.  Therefore, by Lemma~\ref{lem:independence}
        (Independence), we have that
        $\config{\lubstore{S}{\store{\storebindingRaw{l}{u_{p_i}(p_1)}}}}{e_{b_1}}
        \parstepsto
        \config{\lubstore{S_b}{\store{\storebindingRaw{l}{u_{p_i}(p_1)}}}}{e_{b_2}}$.
        By {\sc E-Eval-Ctxt}, it follows that
        $\config{\lubstore{S}{\store{\storebindingRaw{l}{u_{p_i}(p_1)}}}}{\evalctxt{E'_b}{e_{b_1}}}
        \ctxstepsto
        \config{\lubstore{S_b}{\store{\storebindingRaw{l}{u_{p_i}(p_1)}}}}{\evalctxt{E'_b}{e_{b_2}}}$.
        Since
        $\lubstore{S}{\store{\storebindingRaw{l}{u_{p_i}(p_1)}}}
        = \extSRaw{S}{l}{u_{p_i}(p_1)} = S_a$, we have that

        $\config{S_a}{\evalctxt{E'_b}{e_{b_1}}} \ctxstepsto
        \config{\lubstore{S_b}{\store{\storebindingRaw{l}{u_{p_i}(p_1)}}}}{\evalctxt{E'_b}{e_{b_2}}}$.
        Furthermore, since $\config{S}{e_{b_1}} \parstepsto
        \config{S_b}{e_{b_2}}$, by Lemma~\ref{lem:monotonicity}
        (Monotonicity), we have that $\leqstore{S}{S_b}$, so
        $\lubstore{S_b}{\store{\storebindingRaw{l}{u_{p_i}(p_1)}}}
        =
        \lubstore{S_b}{\lubstore{S}{\store{\storebindingRaw{l}{u_{p_i}(p_1)}}}}
        = \lubstore{S_b}{S_a} = \lubstore{S_a}{S_b}$.  So we have that
        $\config{S_a}{\evalctxt{E'_b}{e_{b_1}}} \ctxstepsto
        \config{\lubstore{S_a}{S_b}}{\evalctxt{E'_b}{e_{b_2}}}$, as we
        were required to show.

        The argument for the second is symmetrical, with
        $\store{\storebindingRaw{l'}{\userlub{d'_1}{d'_2}}}$ being the
        transition that is non-conflicting with $\config{S}{e_{a_1}}
        \parstepsto \config{S_a}{e_{a_2}}$.

      \item
        $\lubstore{S_b}{\store{\storebindingRaw{l}{u_{p_i}(p_1)}}}
        = \topS$:

        Here we choose $\conf_c = \error$ and $\pi = \id$.  We have to
        show that there exist $i \leq 1$ and $j \leq 1$ such that:
        \begin{itemize}
        \item $\config{S_a}{\evalctxt{E'_b}{e_{b_1}}} \ctxstepsto^i
          \error$, and
        \item
          $\config{S_b}{\evalctxt{E'_a}{e_{a_1}}} \ctxstepsto^j \error$.
        \end{itemize}

        For the first of these, since no locations are allocated
        during the transition $\config{S}{e_{b_1}} \parstepsto
        \config{S_b}{e_{b_2}}$, we know that
        $\store{\storebindingRaw{l}{u_{p_i}(p_1)}}$ is
        non-conflicting with it, and in this subcase, we know that
        $\lubstore{S_b}{\store{\storebindingRaw{l}{u_{p_i}(p_1)}}}
        = \topS$.  Therefore, by (Clash),
        \TODO{Don't use Clash.}
        we have that
        $\config{\lubstore{S}{\store{\storebindingRaw{l}{u_{p_i}(p_1)}}}}{e_{b_1}}
        \parstepsto^{i'} \error$, where $i' \leq 1$.  Since $\error$ is
        equal to $\config{\topS}{e}$ for all expressions $e$,
        $\config{\lubstore{S}{\store{\storebindingRaw{l}{u_{p_i}(p_1)}}}}{e_{b_1}}
        \parstepsto^{i'} \config{\topS}{e}$ for all $e$.

        Now consider whether $i' = 1$ or $i' = 0$:
        \begin{itemize}
          \item If $i' = 1$, by {\sc E-Eval-Ctxt}, it follows that
            $\config{\lubstore{S}{\store{\storebindingRaw{l}{u_{p_i}(p_1)}}}}{\evalctxt{E'_b}{e_{b_1}}}
            \ctxstepsto \config{\topS}{\evalctxt{E'_b}{e}}$ for all
            $e$.  Since $\config{\topS}{\evalctxt{E'_b}{e}}$ is equal
            to $\error$, and since
            $\lubstore{S}{\store{\storebindingRaw{l}{u_{p_i}(p_1)}}}
            = \extSRaw{S}{l}{u_{p_i}(p_1)} = S_a$, we choose $i
            = 1$ and we have that
            $\config{S_a}{\evalctxt{E'_b}{e_{b_1}}} \ctxstepsto
            \error$, as required.

          \item If $i' = 0$, we have that
            $\config{\lubstore{S}{\store{\storebindingRaw{l}{u_{p_i}(p_1)}}}}{e_{b_1}}
            = \error$.  Hence
            $\lubstore{S}{\store{\storebindingRaw{l}{u_{p_i}(p_1)}}}
            = \topS$.  So, we choose $i = 0$, and since $S_a =
            \extSRaw{S}{l}{u_{p_i}(p_1)} =
            \lubstore{S}{\store{\storebindingRaw{l}{u_{p_i}(p_1)}}}
            = \topS$, we have that
            $\config{S_a}{\evalctxt{E'_b}{e_{b_1}}} = \error$, as
            desired.
        \end{itemize}

        The argument for the second is symmetrical, with
        $\store{\storebindingRaw{l'}{\userlub{d'_1}{d'_2}}}$ being the
        transition that is non-conflicting with $\config{S}{e_{a_1}}
        \parstepsto \config{S_a}{e_{a_2}}$.

      \end{itemize}

    \item \label{slqc-put-put-err}Case {\sc E-Put-Err}:

      Here $\config{S_b}{e_{b_2}} = \error$, and so we choose $\conf_c
      = \error$, $i = 1$, $j = 0$, and $\pi = \id$.  We have to show that:
      \begin{itemize}
      \item $\config{S_a}{\evalctxt{E'_b}{e_{b_1}}} \ctxstepsto
        \error$, and
      \item
        $\config{S_b}{\evalctxt{E'_a}{e_{a_1}}} = \error$.
      \end{itemize}

      The second of these is immediately true because since
      $\config{S_b}{e_{b_2}} = \error$, $S_b = \topS$, and so
      $\config{S_b}{\evalctxt{E'_a}{e_{a_1}}}$ is equal to $\error$ as
      well.  For the first, observe that since $\config{S}{e_{a_1}}
      \parstepsto \config{S_a}{e_{a_2}}$, we have by
      Lemma~\ref{lem:monotonicity} (Monotonicity) that
      $\leqstore{S}{S_a}$.  Therefore, since $\config{S}{e_{b_1}}
      \parstepsto \error$, we have by
      (Error Preservation)
      \TODO{Don't use Error Preservation.}
      that $\config{S_a}{e_{b_1}} \parstepsto \error$.  Since $\error$
      is equal to $\config{\topS}{e}$ for all expressions $e$,
      $\config{S_a}{e_{b_1}} \parstepsto \config{\topS}{e}$ for all
      $e$.  Therefore, by {\sc E-Eval-Ctxt},
      $\config{S_a}{\evalctxt{E'_b}{e_{b_1}}} \ctxstepsto
      \config{\topS}{\evalctxt{E'_b}{e}}$ for all $e$.  Since
      $\config{\topS}{\evalctxt{E'_b}{e}}$ is equal to $\error$, we
      have that $\config{S_a}{\evalctxt{E'_b}{e_{b_1}}} \ctxstepsto
      \error$, as we were required to show.

    \item \label{slqc-put-get}Case {\sc E-Get}: Similar to
      case~\ref{slqc-put-beta}, since $S_b = S$.
    \item \label{slqc-put-freeze-init}Case {\sc E-Freeze-Init}:
      Similar to case~\ref{slqc-put-beta}, since $S_b = S$.
    \item \label{slqc-put-spawn-handler}Case {\sc E-Spawn-Handler}:
      Similar to case~\ref{slqc-put-beta}, since $S_b = S$.
    \item \label{slqc-put-freeze-final}Case {\sc E-Freeze-Final}: \TODO{}
    \item \label{slqc-put-freeze-simple}Case {\sc E-Freeze-Simple}: \TODO{}

    \end{enumerate}
  \item Case {\sc E-Put-Err}: We have $\config{S_a}{e_{a_2}} =
    \error$.

    We proceed by case analysis on the rule by which
    $\config{S}{e_{b_1}}$ steps to $\config{S_b}{e_{b_2}}$:
    \begin{enumerate}
    \item \label{slqc-put-err-beta}Case {\sc E-Beta}: By symmetry with case~\ref{slqc-beta-put-err}.
    \item \label{slqc-put-err-new}Case {\sc E-New}: By symmetry with case~\ref{slqc-new-put-err}.
    \item \label{slqc-put-err-put}Case {\sc E-Put}: By symmetry with case~\ref{slqc-put-put-err}.
    \item \label{slqc-put-err-put-err}Case {\sc E-Put-Err}: 

      Here $\config{S_b}{e_{b_2}} = \error$, and so we choose $\conf_c
      = \error$, $i = 0$, $j = 0$, and $\pi = \id$.  We have to show
      that:
      \begin{itemize}
      \item $\config{S_a}{\evalctxt{E'_b}{e_{b_1}}} = \error$, and
      \item
        $\config{S_b}{\evalctxt{E'_a}{e_{a_1}}} = \error$.
      \end{itemize}

      Since $\config{S_a}{e_{a_2}} = \error$, $S_a = \topS$, and since
      $\config{S_b}{e_{b_2}} = \error$, $S_b = \topS$, so both of the
      above follow immediately.

    \item \label{slqc-put-err-get}Case {\sc E-Get}: Similar to
      case~\ref{slqc-put-err-beta}, since $S_b = S$.
    \item \label{slqc-put-err-freeze-init}Case {\sc E-Freeze-Init}:
      Similar to case~\ref{slqc-put-err-beta}, since $S_b = S$.
    \item \label{slqc-put-err-spawn-handler}Case {\sc E-Spawn-Handler}:
      Similar to case~\ref{slqc-put-err-beta}, since $S_b = S$.
    \item \label{slqc-put-err-freeze-final}Case {\sc E-Freeze-Final}: \TODO{}
    \item \label{slqc-put-err-freeze-simple}Case {\sc E-Freeze-Simple}: \TODO{}

    \end{enumerate}
  \item Case {\sc E-Get}:

    We proceed by case analysis on the rule by which
    $\config{S}{e_{b_1}}$ steps to $\config{S_b}{e_{b_2}}$:
    \begin{enumerate}
    \item \label{slqc-get-beta}Case {\sc E-Beta}: By symmetry with case~\ref{slqc-beta-get}.
    \item \label{slqc-get-new}Case {\sc E-New}: By symmetry with case~\ref{slqc-new-get}.
    \item \label{slqc-get-put}Case {\sc E-Put}: By symmetry with case~\ref{slqc-put-get}.
    \item \label{slqc-get-put-err}Case {\sc E-Put-Err}: By symmetry with case~\ref{slqc-put-err-get}.
    \item \label{slqc-get-get}Case {\sc E-Get}: Similar to
      case~\ref{slqc-get-beta}, since $S_b = S$.
    \item \label{slqc-get-freeze-init}Case {\sc E-Freeze-Init}:
      Similar to case~\ref{slqc-get-beta}, since $S_b = S$.
    \item \label{slqc-get-spawn-handler}Case {\sc E-Spawn-Handler}:
      Similar to case~\ref{slqc-get-beta}, since $S_b = S$.
    \item \label{slqc-get-freeze-final}Case {\sc E-Freeze-Final}: \TODO{}
    \item \label{slqc-get-freeze-simple}Case {\sc E-Freeze-Simple}: \TODO{}
    \end{enumerate}

  \item Case {\sc E-Freeze-Init}:

    We proceed by case analysis on the rule by which
    $\config{S}{e_{b_1}}$ steps to $\config{S_b}{e_{b_2}}$:
    \begin{enumerate}
    \item \label{slqc-freeze-init-beta}Case {\sc E-Beta}: By symmetry with case~\ref{slqc-beta-freeze-init}.
    \item \label{slqc-freeze-init-new}Case {\sc E-New}: By symmetry with case~\ref{slqc-new-freeze-init}.
    \item \label{slqc-freeze-init-put}Case {\sc E-Put}: By symmetry with case~\ref{slqc-put-freeze-init}.
    \item \label{slqc-freeze-init-put-err}Case {\sc E-Put-Err}: By symmetry with case~\ref{slqc-put-err-freeze-init}.
    \item \label{slqc-freeze-init-get}Case {\sc E-Get}: By symmetry with case~\ref{slqc-get-freeze-init}.
    \item \label{slqc-freeze-init-freeze-init}Case {\sc E-Freeze-Init}:
      Similar to case~\ref{slqc-freeze-init-beta}, since $S_b = S$.
    \item \label{slqc-freeze-init-spawn-handler}Case {\sc E-Spawn-Handler}:
      Similar to case~\ref{slqc-freeze-init-beta}, since $S_b = S$.
    \item \label{slqc-freeze-init-freeze-final}Case {\sc E-Freeze-Final}: \TODO{}
    \item \label{slqc-freeze-init-freeze-simple}Case {\sc E-Freeze-Simple}: \TODO{}
    \end{enumerate}

  \item Case {\sc E-Spawn-Handler}:

    We proceed by case analysis on the rule by which
    $\config{S}{e_{b_1}}$ steps to $\config{S_b}{e_{b_2}}$:
    \begin{enumerate}
    \item \label{slqc-spawn-handler-beta}Case {\sc E-Beta}: By symmetry with case~\ref{slqc-beta-spawn-handler}.
    \item \label{slqc-spawn-handler-new}Case {\sc E-New}: By symmetry with case~\ref{slqc-new-spawn-handler}.
    \item \label{slqc-spawn-handler-put}Case {\sc E-Put}: By symmetry with case~\ref{slqc-put-spawn-handler}.
    \item \label{slqc-spawn-handler-put-err}Case {\sc E-Put-Err}: By symmetry with case~\ref{slqc-put-err-spawn-handler}.
    \item \label{slqc-spawn-handler-get}Case {\sc E-Get}: By symmetry with case~\ref{slqc-get-spawn-handler}.
    \item \label{slqc-spawn-handler-freeze-init}Case {\sc E-Freeze-Init}: By symmetry with case~\ref{slqc-freeze-init-spawn-handler}.
    \item \label{slqc-spawn-handler-spawn-handler}Case {\sc E-Spawn-Handler}:
      Similar to case~\ref{slqc-spawn-handler-beta}, since $S_b = S$.
    \item \label{slqc-spawn-handler-freeze-final}Case {\sc E-Freeze-Final}: \TODO{}
    \item \label{slqc-spawn-handler-freeze-simple}Case {\sc E-Freeze-Simple}: \TODO{}
    \end{enumerate}

  \item Case {\sc E-Freeze-Final}:

    We proceed by case analysis on the rule by which
    $\config{S}{e_{b_1}}$ steps to $\config{S_b}{e_{b_2}}$:
    \begin{enumerate}
    \item \label{slqc-freeze-final-beta}Case {\sc E-Beta}: By symmetry with case~\ref{slqc-beta-freeze-final}.
    \item \label{slqc-freeze-final-new}Case {\sc E-New}: By symmetry with case~\ref{slqc-new-freeze-final}.
    \item \label{slqc-freeze-final-put}Case {\sc E-Put}: By symmetry with case~\ref{slqc-put-freeze-final}.
    \item \label{slqc-freeze-final-put-err}Case {\sc E-Put-Err}: By symmetry with case~\ref{slqc-put-err-freeze-final}.
    \item \label{slqc-freeze-final-get}Case {\sc E-Get}: By symmetry with case~\ref{slqc-get-freeze-final}.
    \item \label{slqc-freeze-final-freeze-init}Case {\sc E-Freeze-Init}: By symmetry with case~\ref{slqc-freeze-init-freeze-final}.
    \item \label{slqc-freeze-final-spawn-handler}Case {\sc E-Spawn-Handler}: By symmetry with case~\ref{slqc-spawn-handler-freeze-final}.
    \item \label{slqc-freeze-final-freeze-final}Case {\sc E-Freeze-Final}: \TODO{}
    \item \label{slqc-freeze-final-freeze-simple}Case {\sc E-Freeze-Simple}: \TODO{}
    \end{enumerate}

  \item Case {\sc E-Freeze-Simple}:

    \begin{enumerate}
    \item \label{slqc-freeze-simple-beta}Case {\sc E-Beta}: By symmetry with case~\ref{slqc-beta-freeze-simple}.
    \item \label{slqc-freeze-simple-new}Case {\sc E-New}: By symmetry with case~\ref{slqc-new-freeze-simple}.
    \item \label{slqc-freeze-simple-put}Case {\sc E-Put}: By symmetry with case~\ref{slqc-put-freeze-simple}.
    \item \label{slqc-freeze-simple-put-err}Case {\sc E-Put-Err}: By symmetry with case~\ref{slqc-put-err-freeze-simple}.
    \item \label{slqc-freeze-simple-get}Case {\sc E-Get}: By symmetry with case~\ref{slqc-get-freeze-simple}.
    \item \label{slqc-freeze-simple-freeze-init}Case {\sc E-Freeze-Init}: By symmetry with case~\ref{slqc-freeze-init-freeze-simple}.
    \item \label{slqc-freeze-simple-spawn-handler}Case {\sc E-Spawn-Handler}: By symmetry with case~\ref{slqc-spawn-handler-freeze-simple}.
    \item \label{slqc-freeze-simple-freeze-final}Case {\sc E-Freeze-Final}: By symmetry with case~\ref{slqc-freeze-final-freeze-simple}.
    \item \label{slqc-freeze-simple-freeze-simple}Case {\sc E-Freeze-Simple}: \TODO{}
    \end{enumerate}

  \end{enumerate}
\end{proof}

