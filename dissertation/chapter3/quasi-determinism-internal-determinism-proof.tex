\begin{proof}
  Suppose $\conf \parstepsto \conf'$ and $\conf \parstepsto \conf''$.
  We have to show that there is a permutation $\pi$ such that $\conf'
  = \pi(\conf'')$, modulo choice of events.  The proof is by cases on
  the rule by which $\conf$ steps to $\conf'$.

  \begin{itemize}

  \item Case {\sc E-Beta}:

    Given: $\config{S}{\app{(\lam{x}{e})}{v}} \parstepsto
    \config{S}{\subst{e}{x}{v}}$, and
    $\config{S}{\app{(\lam{x}{e})}{v}} \parstepsto \conf''$.

    To show: There exists a $\pi$ such that
    $\config{S}{\subst{e}{x}{v}} = \pi(\conf'')$.

    By inspection of the operational semantics, the only reduction
    rule by which $\config{S}{\app{(\lam{x}{e})}{v}}$ can step is
    {\sc E-Beta}.  Hence $\conf'' = \config{S}{\subst{e}{x}{v}}$,
    and the case is satisfied by choosing $\pi$ to be the identity
    function.

  \item Case {\sc E-New}:

    Given: $\config{S}{\NEW} \parstepsto
    \config{\extS{S}{l}{\bot}{\frozenfalse}}{l}$, and
    $\config{S}{\NEW} \parstepsto \conf''$.

    To show: There exists a $\pi$ such that
    $\config{\extS{S}{l}{\bot}{\frozenfalse}}{l} = \pi(\conf'')$.

    By inspection of the operational semantics, the only reduction
    rule by which $\config{S}{\NEW}$ can step is {\sc E-New}.  Hence
    $\conf'' = \config{\extS{S}{l'}{\bot}{\frozenfalse}}{l'}$.  Since,
    by the side condition of {\sc E-New}, neither $l$ nor $l'$ occur
    in $\dom{S}$, the case is satisfied by choosing $\pi$ to be the
    permutation that maps $l'$ to $l$ and is the identity on every
    other element of $\Loc$.

  \item Case {\sc E-Put}:

    Given: $\config{S}{\putiexp{l}} \parstepsto
    \config{\extSRaw{S}{l}{u_{p_i}(p_1)}}{\unit}$, and
    $\config{S}{\putiexp{l}} \parstepsto \conf''$.

    To show: There exists a $\pi$ such that
    $\config{\extSRaw{S}{l}{u_{p_i}(p_1)}}{\unit} = \pi(\conf'')$.

    By inspection of the operational semantics, and since
    $u_{p_i}(p_1) \neq \topp$ (from the premise of {\sc E-Put}), the
    only reduction rule by which $\config{S}{\putiexp{l}}$ can step is
    {\sc E-Put}.  Hence $\conf'' =
    \config{\extSRaw{S}{l}{u_{p_i}(p_1)}}{\unit}$, and the case is
    satisfied by choosing $\pi$ to be the identity function.

  \item Case {\sc E-Put-Err}:

    Given: $\config{S}{\putiexp{l}} \parstepsto \error$, and
    $\config{S}{\putiexp{l}} \parstepsto \conf''$.

    To show: There exists a $\pi$ such that $\error = \pi(\conf'')$.
    By inspection of the operational semantics, and since
    $u_{p_i}(p_1) = \topp$ (from the premise of {\sc E-Put-Err}), the
    only reduction rule by which $\config{S}{\putiexp{l}}$ can step is
    {\sc E-Put-Err}.  Hence $\conf'' = \error$, and the case is
    satisfied by choosing $\pi$ to be the identity function.

  \item Case {\sc E-Get}:

    Given: $\config{S}{\getexp{l}{P}} \parstepsto \config{S}{p_2}$,
    and $\config{S}{\getexp{l}{P}} \parstepsto \conf''$.

    To show: There exists a $\pi$ such that $\config{S}{p_2} =
    \pi(\conf'')$.

    By inspection of the operational semantics, the only reduction
    rule by which $\config{S}{\getexp{l}{P}}$ can step is {\sc E-Get}.
    Hence $\conf'' = \config{S}{p_2}$, and the case is satisfied by
    choosing $\pi$ to be the identity function.

  \item Case {\sc E-Freeze-Init}:

    Given: $\config{S}{\freezeafter{l}{Q}{\lam{x}{e}}} \parstepsto
    \config{S}{\freezeafterfull{l}{Q}{\lam{x}{e}}{\setof{}}{\setof{}}}$,
    and $\config{S}{\freezeafter{l}{Q}{\lam{x}{e}}} \parstepsto \conf''$.

    To show: There exists a $\pi$ such that
    $\config{S}{\freezeafterfull{l}{Q}{\lam{x}{e}}{\setof{}}{\setof{}}}
    = \pi(\conf'').$

    By inspection of the operational semantics, the only reduction
    rule by which $\config{S}{\freezeafter{l}{Q}{\lam{x}{e}}}$ can
    step is {\sc E-Freeze-Init}.  Hence $\conf'' =
    \config{S}{\freezeafterfull{l}{Q}{\lam{x}{e}}{\setof{}}{\setof{}}}$,
    and the case is satisfied by choosing $\pi$ to be the identity
    function.

  \item Case {\sc E-Spawn-Handler}:

    Given: $\config{S}{\freezeafterfull{l}{Q}{\lam{x}{e_0}}{\setof{e,
          \dots}}{H}} \parstepsto
    \config{S}{\freezeafterfull{l}{Q}{\lam{x}{e_0}}{\setof{\subst{e_0}{x}{d_2},
          e, \dots}} {\{d_2\}\cup H}}$, and
    $\config{S}{\freezeafterfull{l}{Q}{\lam{x}{e_0}}{\setof{e,
          \dots}}{H}} \parstepsto \conf''$.

    To show: There exists a $\pi$ such that
    $\config{S}{\freezeafterfull{l}{Q}{\lam{x}{e_0}}{\setof{\subst{e_0}{x}{d_2},
          e, \dots}} {\{d_2\}\cup H}} = \pi(\conf'').$

    By inspection of the operational semantics, the only reduction
    rule by which
    $\config{S}{\freezeafterfull{l}{Q}{\lam{x}{e_0}}{\setof{e,
          \dots}}{H}}$ can step is {\sc E-Spawn-Handler}.  (It cannot
    step by {\sc E-Freeze-Final}, because we have from the premises of
    {\sc E-Spawn-Handler} that $d_2 \userleq d_1$ and $d_2 \in Q$ and
    $d_2 \notin H$, and for the premises of {\sc E-Freeze-Final} to
    hold, we would need that for all $d_2$, if $d_2 \userleq d_1$ and
    $d_2 \in Q$, then $d_2 \in H$.)  Hence $\conf'' =
    \config{S}{\freezeafterfull{l}{Q}{\lam{x}{e_0}}{\setof{\subst{e_0}{x}{d'_2},
          e, \dots}} {\{d'_2\}\cup H}}$, where $d'_2 \userleq d_1$ and
    $d'_2 \in Q$ and $d'_2 \notin H$, and the case is satisfied by
    choosing $\pi$ to be the identity function.  (It may be the case
    that $d'_2 \neq d_2$; if so, then we have internal nondeterminism
    modulo \emph{choice of events}, as we were required to show.  If
    $d'_2 = d_2$ then we have internal nondeterminism even without
    that additional qualification, which also satisfies the case.)

  \item Case {\sc E-Freeze-Final}:

    Given: $\config{S}{\freezeafterfull{l}{Q}{\lam{x}{e_0}}{\setof{v,
          \dots}}{H}} \parstepsto
    \config{\extS{S}{l}{d_1}{\frozentrue}}{d_1}$, and
    $\config{S}{\freezeafterfull{l}{Q}{\lam{x}{e_0}}{\setof{v,
          \dots}}{H}} \parstepsto \conf''$.

    To show: There exists a $\pi$ such that
    $\config{\extS{S}{l}{d_1}{\frozentrue}}{d_1} = \pi(\conf'').$

    By inspection of the operational semantics, the only reduction
    rule by which
    $\config{S}{\freezeafterfull{l}{Q}{\lam{x}{e_0}}{\setof{v,
          \dots}}{H}}$ can step is {\sc E-Freeze-Final}.  (It cannot
    step by {\sc E-Spawn-Handler}, because we have from the premises
    of {\sc E-Freeze-Final} that, for all $d_2$, if $d_2 \userleq d_1$
    and $d_2 \in Q$, then $d_2 \in H$, and for the premises of {\sc
      E-Spawn-Handler} to hold, we would need that $d_2 \userleq d_1$
    and $d_2 \in Q$ and $d_2 \notin H$.)  Hence $\conf'' =
    \config{\extS{S}{l}{d_1}{\frozentrue}}{d_1}$, and the case is
    satisfied by choosing $\pi$ to be the identity function.

  \item Case {\sc E-Freeze-Simple}:

    Given: $\config{S}{\freeze{l}} \parstepsto
    \config{\extS{S}{l}{d_1}{\frozentrue}}{d_1}$, and
    $\config{S}{\freeze{l}} \parstepsto \conf''$.

    To show: There exists a $\pi$ such that
    $\config{\extS{S}{l}{d_1}{\frozentrue}}{d_1} = \pi(\conf'').$

    By inspection of the operational semantics, the only reduction
    rule by which $\config{S}{\freeze{l}}$ can step is {\sc
      E-Freeze-Simple}.  Hence $\conf'' =
    \config{\extS{S}{l}{d_1}{\frozentrue}}{d_1}$, and the case is
    satisfied by choosing $\pi$ to be the identity function.

  \end{itemize}
\end{proof}

