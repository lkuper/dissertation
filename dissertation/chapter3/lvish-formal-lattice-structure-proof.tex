\begin{proof}
~\begin{enumerate}
\item $\leqp$ is a partial order over $D_p$.

To show this, we need to show that $\leqp$ is reflexive, transitive,
and antisymmetric. 
\begin{enumerate}
\item $\leqp$ is reflexive.

  Suppose $v \in D_p$. Then, by Lemma~\ref{lem:partition-of-Dp},
  either $v = \state{d}{\frozenfalse}$ with $d \in D$, or $v = \state{x}{\frozentrue}$ with $x \in X$, where $X = D - \setof{\top}$.
  \begin{itemize}
  \item Suppose $v = \state{d}{\frozenfalse}$:

    By the reflexivity of $\userleq$, we know $d \userleq d$. \\ 
    By the definition of $\leqp$, we know $\state{d}{\frozenfalse} \leqp \state{d}{\frozenfalse}$.

  \item Suppose $v = \state{x}{\frozentrue}$: 
   
    By the reflexivity of equality, $x = x$. \\ 
    By the definition of $\leqp$, we know $\state{x}{\frozentrue} \leqp \state{x}{\frozentrue}$. 
  \end{itemize}

\item $\leqp$ is transitive. 

  Suppose $v_1 \leqp v_2$ and $v_2 \leqp v_3$. We want to show that $v_1 \leqp v_3$. We
  proceed by case analysis on $v_1, v_2$, and $v_3$. 
  \begin{itemize}
  \item Case $v_1 = \state{d_1}{\frozenfalse}$ and $v_2 = \state{d_2}{\frozenfalse}$ and $v_3 = \state{d_3}{\frozenfalse}$:
    
    By inversion on $\leqp$, it follows that $d_1 \userleq d_2$. \\ 
    By inversion on $\leqp$, it follows that $d_2 \userleq d_3$. \\ 
    By the transitivity of $\userleq$, we know $d_1 \userleq d_3$. \\ 
    By the definition of $\leqp$, it follows that $\state{d_1}{\frozenfalse} \leqp \state{d_3}{\frozenfalse}$. \\ 
    Hence $v_1 \leqp v_3$. 

  \item Case $v_1 = \state{d_1}{\frozenfalse}$ and $v_2 = \state{d_2}{\frozenfalse}$ and $v_3 = \state{x_3}{\frozentrue}$:

    By inversion on $\leqp$, it follows that $d_1 \userleq d_2$. \\ 
    By inversion on $\leqp$, it follows that $d_2 \userleq x_3$. \\ 
    By the transitivity of $\userleq$, we know $d_1 \userleq x_3$. \\ 
    By the definition of $\leqp$, it follows that $\state{d_1}{\frozenfalse} \leqp \state{x_3}{\frozentrue}$. \\ 
    Hence $v_1 \leqp v_3$. 

  \item Case $v_1 = \state{d_1}{\frozenfalse}$ and $v_2 = \state{x_2}{\frozentrue}$ and $v_3 = \state{d_3}{\frozenfalse}$:

    By inversion on $\leqp$, it follows that $d_1 \userleq x_2$. \\ 
    By inversion on $\leqp$, it follows that $d_3 = \top$. \\ 
    Since $\top$ is the maximal element of $D$, we know $d_1 \userleq \top \equiv d_3$. \\ 
    By the definition of $\leqp$, it follows that $\state{d_1}{\frozenfalse} \leqp \state{d_3}{\frozenfalse}$. \\ 
    Hence $v_1 \leqp v_3$. 

  \item Case $v_1 = \state{d_1}{\frozenfalse}$ and $v_2 = \state{x_2}{\frozentrue}$ and $v_3 = \state{x_3}{\frozentrue}$:

    By inversion on $\leqp$, it follows that $d_1 \userleq x_2$. \\ 
    By inversion on $\leqp$, it follows that $x_2 = x_3$. \\ 
    Hence $d_1 \userleq x_3$. \\ 
    By the definition of $\leqp$, it follows that $\state{d_1}{\frozenfalse} \leqp \state{x_3}{\frozentrue}$. \\ 
    Hence $v_1 \leqp v_3$. 

  \item Case $v_1 = \state{x_1}{\frozentrue}$ and $v_2 = \state{d_2}{\frozenfalse}$ and $v_3 = \state{d_3}{\frozenfalse}$:

    By inversion on $\leqp$, it follows that $d_2 = \top$. \\ 
    By inversion on $\leqp$, it follows that $d_2 \userleq d_3$. \\ 
    Since $\top$ is maximal, it follows that $d_3 = \top$. \\ 
    By the definition of $\leqp$, it follows that $\state{x_1}{\frozentrue} \leqp \state{d_3}{\frozenfalse}$. \\ 
    Hence $v_1 \leqp v_3$. 

  \item Case $v_1 = \state{x_1}{\frozentrue}$ and $v_2 = \state{d_2}{\frozenfalse}$ and $v_3 = \state{x_3}{\frozentrue}$:

    By inversion on $\leqp$, it follows that $d_2 = \top$. \\ 
    By inversion on $\leqp$, it follows that $d_2 \userleq x_3$. \\ 
    Since $\top$ is maximal, it follows that $x_3 = \top$. \\
    But since $x_3 \in X \subseteq D/\setof{\top}$, we know $x_3 \not= \top$. \\ 
    This is a contradiction. \\
    Hence $v_1 \leqp v_3$. 


  \item Case $v_1 = \state{x_1}{\frozentrue}$ and $v_2 = \state{x_2}{\frozentrue}$ and $v_3 = \state{d_3}{\frozenfalse}$:

    By inversion on $\leqp$, it follows that $x_1 = x_2$. \\ 
    By inversion on $\leqp$, it follows that $d_3 = \top$. \\ 
    By the definition of $\leqp$, it follows that $\state{x_1}{\frozentrue} \leqp \state{d_3}{\frozenfalse}$. \\ 
    Hence $v_1 \leqp v_3$. 

  \item Case $v_1 = \state{x_1}{\frozentrue}$ and $v_2 = \state{x_2}{\frozentrue}$ and $v_3 = \state{x_3}{\frozentrue}$:

    By inversion on $\leqp$, it follows that $x_1 = x_2$. \\ 
    By inversion on $\leqp$, it follows that $x_2 = x_3$. \\ 
    By transitivity of $=$, $x_1 = x_3$. \\ 
    By the definition of $\leqp$, it follows that $\state{x_1}{\frozentrue} \leqp \state{x_3}{\frozentrue}$. \\ 
    Hence $v_1 \leqp v_3$. 
    
  \end{itemize}

\item $\leqp$ is antisymmetric. 

  Suppose $v_1 \leqp v_2$ and $v_2 \leqp v_1$. Now, we proceed by cases on $v_1$ and $v_2$. 
  \begin{itemize}
  \item Case $v_1 = \state{d_1}{\frozenfalse}$ and $v_2 = \state{d_2}{\frozenfalse}$: 
    
    By inversion on $v_1 \leqp v_2$, we know that $d_1 \userleq d_2$. \\ 
    By inversion on $v_2 \leqp v_1$, we know that $d_2 \userleq d_1$. \\ 
    By the antisymmetry of $\leq$, we know $d_1 = d_2$. \\ 
    Hence $v_1 = v_2$. 

  \item Case $v_1 = \state{d_1}{\frozenfalse}$ and $v_2 = \state{x_2}{\frozentrue}$: 

    By inversion on $v_1 \leqp v_2$, we know that $d_1 \userleq x_2$. \\ 
    By inversion on $v_2 \leqp v_1$, we know that $d_1 = \top$. \\ 
    Since $\top$ is maximal in $D$, we know $x_2 = \top$. \\ 
    But since $x_2 \in X \subseteq D/\setof{\top}$, we know $x_2 \not= \top$. \\ 
    This is a contradiction. \\
    Hence $v_1 = v_2$. 
    
  \item Case $v_1 = \state{x_1}{\frozentrue}$ and $v_2 = \state{d_2}{\frozenfalse}$: 

    Similar to the previous case. 

  \item Case $v_1 = \state{x_1}{\frozentrue}$ and $v_2 = \state{x_2}{\frozentrue}$: 

    By inversion on $v_1 \leqp v_2$, we know that $x_1 = x_2$. \\
    Hence $v_1 = v_2$. 
  \end{itemize}
\end{enumerate}

\item Every nonempty finite subset of $D_p$ has a least upper bound.

To show this, it is sufficient to show that every two elements of
$D_p$ have a least upper bound, since a binary least upper bound
operation can be repeatedly applied to compute the least upper bound
of any finite set.  We will show that every two elements of $D_p$ have
a least upper bound by showing that the $\lubp{}{}$ operation defined
by Definition~\ref{def:lubp} computes their least upper bound.

%% $\lubp{v_1}{v_2}$ is the least upper bound of $v_1$ and $v_2$ if it is
%% an element $u$ of $D_p$ such that $v_1 \leqp u$, $v_2 \leqp u$, and
%% for any $v \in D_p$ such that $v_1 \leqp v$ and $v_2 \leqp v$, it
%% holds that $u \leqp v$.

It suffices to show the following two properties:
  \begin{enumerate}
  \item For all $v_1, v_2, v \in D_p$, if $v_1 \leqp v$ and $v_2 \leqp v$, then $(\lubp{v_1}{v_2}) \leqp v$.
  \item For all $v_1, v_2 \in D_p$, $v_1 \leqp (\lubp{v_1}{v_2})$ and $v_2 \leqp (\lubp{v_1}{v_2})$. 
  \end{enumerate}
  \begin{enumerate}
  \item For all $v_1, v_2, v \in D_p$, if $v_1 \leqp v$ and $v_2 \leqp v$, then $\lubp{v_1}{v_2} \leqp v$.
    
   Assume $v_1, v_2, v \in D_p$, and $v_1 \leqp v$ and $v_2 \leqp v$. Now we do a case analysis on 
   $v_1$ and $v_2$. 
   \begin{itemize}
   \item Case $v_1 = \state{d_1}{\frozenfalse}$ and $v_2 = \state{d_2}{\frozenfalse}$. 
  
     Now case on $v$: 
     \begin{itemize}
     \item Case $v = \state{d}{\frozenfalse}$: 
  
       By the definition of $\lubp{}{}$, $\lubp{\state{d_1}{\frozenfalse}}{\state{d_2}{\frozenfalse}} = \state{\userlub{d_1}{d_2}}{\frozenfalse}$. \\ 
       By inversion on $\state{d_1}{\frozenfalse} \leqp \state{d}{\frozenfalse}$,  $d_1 \userleq l$. \\
       By inversion on $\state{d_2}{\frozenfalse} \leqp \state{d}{\frozenfalse}$,  $d_2 \userleq l$. \\
       Hence $l$ is an upper bound for $d_1$ and $d_2$. \\ 
       Hence $\userlub{d_1}{d_2} \userleq l$. \\ 
       Hence $\state{\userlub{d_1}{d_2}}{\frozenfalse} \leqp \state{d}{\frozenfalse}$. \\ 
       Hence $\lubp{v_1}{v_2} \leqp v$. 
  
     \item Case $v = \state{x}{\frozentrue}$: 
  
       By the definition of $\lubp{}{}$, $\state{d_1}{\frozenfalse} \lubp{}{} \state{d_2}{\frozenfalse} = \state{\userlub{d_1}{d_2}}{\frozenfalse}$. \\ 
       By inversion on $\state{d_1}{\frozenfalse} \leqp \state{x}{\frozentrue}$,  $d_1 \userleq x$. \\
       By inversion on $\state{d_2}{\frozenfalse} \leqp \state{x}{\frozentrue}$,  $d_2 \userleq x$. \\
       Hence $x$ is an upper bound for $d_1$ and $d_2$. \\ 
       Hence $\userlub{d_1}{d_2} \userleq x$. \\ 
       Hence $\state{\userlub{d_1}{d_2}}{\frozenfalse} \leqp \state{x}{\frozentrue}$. \\ 
       Hence $\lubp{v_1}{v_2} \leqp v$. 
     \end{itemize}
  
   \item Case $v_1 = \state{x_1}{\frozentrue}$ and $v_2 = \state{x_2}{\frozentrue}$: 
  
     Now case on $v$: 
     \begin{itemize}
     \item Case $v = \state{d}{\frozenfalse}$: 
  
       By inversion on $\state{x_1}{\frozentrue} \leqp \state{d}{\frozenfalse}$, we know $l = \top$. \\
       By inversion on $\state{x_2}{\frozentrue} \leqp \state{d}{\frozenfalse}$, we know $l = \top$. \\
       Now consider whether $x_1 = x_2$ or not. 
       If it does, then  by the definition of $\lubp{}{}$, $\state{x_1}{\frozentrue} \lubp{}{} \state{x_2}{\frozentrue} = \state{x_1}{\frozentrue}$. \\ 
       By definition of $\leqp$, we have $\state{x_1}{\frozentrue} \leqp \state{\top}{\frozenfalse}$. 
       So $\lubp{v_1}{v_2} \leqp v$. \\ 
       If it does not, then $\lubp{v_1}{v_2} = \state{\top}{\frozenfalse}$. \\ 
       By the definition of $\leqp$, we have $\state{\top}{\frozenfalse} \leqp \state{\top}{\frozenfalse}$. 
       So $\lubp{v_1}{v_2} \leqp v$. 
  
     \item Case $v = \state{x}{\frozentrue}$: 
  
       By inversion on $\state{x_1}{\frozentrue} \leqp \state{x}{\frozentrue}$, we know $x = x_1$. \\
       By inversion on $\state{x_2}{\frozentrue} \leqp \state{x}{\frozentrue}$, we know $x = x_2$. \\
       Hence $x_1 = x_2$. \\ 
       By the definition of $\lubp{}{}$, $\state{x_1}{\frozentrue} \lubp{}{} \state{x_2}{\frozentrue} = \state{x_1}{\frozentrue}$. \\
       Hence $\lubp{v_1}{v_2} \leqp v$. 
     \end{itemize}
   
   \item Case $v_1 = \state{x_1}{\frozentrue}$ and $v_2 = \state{d_2}{\frozenfalse}$: 
  
     Now case on $v$:
     \begin{itemize}
     \item Case $v = \state{d}{\frozenfalse}$: 
     
       Now consider whether $d_2 \userleq x_1$. \\
       If it is, then $\state{x_1}{\frozentrue} \lubp{}{} \state{d_2}{\frozenfalse} = \state{x_1}{\frozentrue} = v_1$. \\ 
       Hence $\lubp{v_1}{v_2} \leqp v$. \\ 
       Otherwise, $\state{x_1}{\frozentrue} \lubp{}{} \state{d_2}{\frozenfalse} = \state{\top}{\frozenfalse}$. \\ 
       By inversion on $\state{x_1}{\frozentrue} \leqp \state{d}{\frozenfalse}$, we know $l = \top$. \\
       By reflexivity, $\state{\top}{\frozenfalse} \leqp \state{\top}{\frozenfalse}$. \\ 
       Hence $\lubp{v_1}{v_2} \leqp v$. 
       
     \item Case $v = \state{x}{\frozentrue}$:  
  
       By inversion on $\state{x_1}{\frozentrue} \leqp \state{x}{\frozentrue}$, we know that $x_1 = x$. \\ 
       By inversion on $\state{d_2}{\frozenfalse} \leqp \state{x}{\frozentrue}$, we know that $d_2 \userleq x$. \\ 
       By transitivity, $d_2 \userleq x_1$. \\ 
       By the definition of $\lubp{}{}$, it follows that $\state{x_1}{\frozentrue} \lubp{}{} \state{d_2}{\frozenfalse} = \state{x_1}{\frozentrue}$. \\ 
       By definition of $\leqp$, $\state{x_1}{\frozentrue} \leqp \state{x_1}{\frozentrue}$. \\ 
       Hence $\lubp{v_1}{v_2} \leqp v$. 
     \end{itemize}
  
   \item Case $v_1 = \state{d_1}{\frozenfalse}$ and $v_2 = \state{x_2}{\frozentrue}$: 
  
     Symmetric with the previous case. 
   \end{itemize}
 \item For all $v_1, v_2 \in D_p$, $v_1 \leqp \lubp{v_1}{v_2}$ and $v_2 \leqp \lubp{v_1}{v_2}$.
    
    Assume $v_1, v_2 \in D_p$, and proceed by case analysis. 
    \begin{itemize}
      \item Case $v_1 = \state{d_1}{\frozenfalse}$ and $v_2 = \state{d_2}{\frozenfalse}$:

        Since $\userlub{}{}$ is a join operator, we know $d_1 \userleq \userlub{d_1}{d_2}$.\\
        By the definition of $\leqp$, $\state{d_1}{\frozenfalse} \userleq \state{\userlub{d_1}{d_2}}{\frozenfalse}$.\\
        By the definition of $\lubp{}{}$, $\lubp{v_1}{v_2} = \state{\userlub{d_1}{d_2}}{\frozenfalse}$.\\
        Hence $v_1 \leqp \lubp{v_1}{v_2}$.  \\

        Since $\userlub{}{}$ is a join operator, we know $d_1 \userleq \userlub{d_1}{d_2}$.\\
        By the definition of $\leqp$, $\state{d_2}{\frozenfalse} \userleq \state{\userlub{d_1}{d_2}}{\frozenfalse}$.\\
        By the definition of $\lubp{}{}$, $\lubp{v_1}{v_2} = \state{\userlub{d_1}{d_2}}{\frozenfalse}$.\\
        Hence $v_2 \leqp \lubp{v_1}{v_2}$. 

        Therefore $v_1 \leqp v_1 \userlub{}{} v_2$ and $v_2 \leqp v_1 \userlub{}{} v_2$. 
      \item Case $v_1 = \state{d_1}{\frozenfalse}$ and $v_2 = \state{x_2}{\frozentrue}$:

        Consider whether $d_1 \userleq x_2$. 
        \begin{itemize}
          \item Case  $d_1 \userleq x_2$:

            By the definition of $\lubp{}{}$, we know $\state{d_1}{\frozenfalse} \lubp{}{} \state{x_2}{\frozentrue} = \state{x_2}{\frozentrue}$. \\ 
            By the definition of $\lubp{}{}$, we know $\state{d_1}{\frozenfalse} \leqp \state{x_2}{\frozentrue}$. \\ 
            Hence $v_1 \leqp \lubp{v_1}{v_2}$. \\ 
            By reflexivity, $\state{x_2}{\frozentrue} \leqp \state{x_2}{\frozentrue}$. \\ 
            Hence $v_2 \leqp \lubp{v_1}{v_2}$. \\ 
            Therefore $v_1 \leqp v_1 \userlub{}{} v_2$ and $v_2 \leqp v_1 \userlub{}{} v_2$. 

          \item Case $d_1 \not\userleq x_2$:

            By the definition of $\lubp{}{}$, we know $\state{d_1}{\frozenfalse} \lubp{}{} \state{x_2}{\frozentrue} = \state{\top}{\frozenfalse}$. \\ 
            Since $d_1 \userleq \top$, by the definition of $\leqp$ we know $\state{d_1}{\frozenfalse} \userleq \state{\top}{\frozenfalse}$. \\ 
            Hence $v_1 \leqp \lubp{v_1}{v_2}$. \\ 
            By the definition of $\leqp$, we know $\state{x_2}{\frozentrue} \userleq \state{\top}{\frozenfalse}$. \\ 
            Hence $v_2 \leqp \lubp{v_1}{v_2}$. \\ 
            Therefore $v_1 \leqp v_1 \userlub{}{} v_2$ and $v_2 \leqp v_1 \userlub{}{} v_2$. 
          \end{itemize}
        \item Case $v_1 = \state{x_1}{\frozentrue}$ and $v_2 = \state{d_2}{\frozenfalse}$: 

          Symmetric with the previous case. 
        \item Case $v_1 = \state{x_1}{\frozentrue}$ and $v_2 = \state{x_2}{\frozentrue}$:

          Consider whether $x_1$ equals $x_2$. 
          \begin{itemize}
            \item Case $x_1 = x_2$:
  
              By the definition $\lubp{}{}$, $\state{x_1}{\frozentrue} \lubp{}{} \state{x_2}{\frozentrue} = \state{x_1}{\frozentrue}$. \\ 
              By reflexivity, $\state{x_1}{\frozentrue} \leqp \state{x_1}{\frozentrue}$. \\ 
              Hence $v_1 \leqp \lubp{v_1}{v_2}$. \\ 
              By reflexivity, $\state{x_2}{\frozentrue} \leqp \state{x_1}{\frozentrue}$. \\ 
              Hence $v_2 \leqp \lubp{v_1}{v_2}$. \\ 
              Therefore $v_1 \leqp v_1 \userlub{}{} v_2$ and $v_2 \leqp v_1 \userlub{}{} v_2$. 

            \item Case $x_1 \not= x_2$: 

              By the definition $\lubp{}{}$, $\state{x_1}{\frozentrue} \lubp{}{} \state{x_2}{\frozentrue} = \state{\top}{\frozenfalse}$. \\ 
              By the definition of $\leqp$, $\state{x_1}{\frozentrue} \leqp \state{\top}{\frozenfalse}$. \\
              Hence $v_1 \leqp \lubp{v_1}{v_2}$. \\ 
              By the definition of $\leqp$, $\state{x_2}{\frozentrue} \leqp \state{\top}{\frozenfalse}$. \\
              Hence $v_2 \leqp \lubp{v_1}{v_2}$. \\ 
              Therefore $v_1 \leqp v_1 \userlub{}{} v_2$ and $v_2 \leqp v_1 \userlub{}{} v_2$. 
          \end{itemize}
    \end{itemize}
  \end{enumerate}

\item $\botp$ is the least element of $D_p$. 

$\botp$ is defined to be $\state{\bot}{\frozenfalse}$.  In order to be the
  least element of $D_p$, it must be less than or equal to every
  element of $D_p$.  By Lemma~\ref{lem:partition-of-Dp}, the elements
  of $D_p$ partition into $\state{d}{\frozenfalse}$ for all $d \in D$, and
  $\state{x}{\frozentrue}$ for all $x \in X$, where $X = D - \setof{\top}$.

  We consider both cases:

  \begin{itemize}
  \item $\state{d}{\frozenfalse}$ for all $d \in D$:

  By the definition of $\leqp$, $\state{\bot}{\frozenfalse} \leqp \state{d}{\frozenfalse}$ iff $\bot \userleq d$. \\
  Since $\bot$ is the least element of $D$, $\bot \userleq d$. \\
  Therefore $\botp = \state{\bot}{\frozenfalse} \leqp \state{d}{\frozenfalse}$.

  \item $\state{x}{\frozentrue}$ for all $x \in X$:

  By the definition of $\leqp$, $\state{\bot}{\frozenfalse} \leqp \state{x}{\frozentrue}$ iff $\bot \userleq x$. \\
  Since $\bot$ is the least element of $D$, $\bot \userleq x$. \\
  Therefore $\botp = \state{\bot}{\frozenfalse} \leqp \state{x}{\frozentrue}$.

  \end{itemize}

  Therefore $\botp$ is less than or equal to all elements of $D_p$.

\item $\topp$ is the greatest element of $D_p$.

$\topp$ is defined to be $\state{\top}{\frozenfalse}$.  In order to be the
  greatest element of $D_p$, every element of $D_p$ must be less than
  or equal to it. By Lemma~\ref{lem:partition-of-Dp}, the elements of
  $D_p$ partition into $\state{d}{\frozenfalse}$ for all $d \in D$, and $\state{x}{\frozentrue}$ for all $x \in X$, where $X = D - \setof{\top}$.

  We consider both cases:

  \begin{itemize}
  \item $\state{d}{\frozenfalse}$ for all $d \in D$:

  By the definition of $\leqp$, $\state{d}{\frozenfalse} \leqp \state{\top}{\frozenfalse}$ iff $d \userleq \top$. \\
  Since $\top$ is the greatest element of $D$, $d \userleq \top$. \\
  Therefore $\state{d}{\frozenfalse} \leqp \state{\top}{\frozenfalse} = \topp$.

  \item $\state{x}{\frozentrue}$ for all $x \in X$:

  By the definition of $\leqp$, $\state{x}{\frozentrue} \leqp \state{\top}{\frozenfalse}$ iff $\top \userleq \top$. \\
  Therefore $\state{x}{\frozentrue} \leqp \state{\top}{\frozenfalse} = \topp$.

  \end{itemize}

  Therefore all elements of $D_p$ are less than or equal to $\topp$.
\end{enumerate}
\end{proof}
