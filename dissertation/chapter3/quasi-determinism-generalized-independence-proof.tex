\begin{proof}
  Suppose $\config{S}{e} \parstepsto \config{S'}{e'}$, where
  $\config{S'}{e'} \neq \error$.

  Consider arbitrary $\Ustore$ such that $\Ustore$ is non-conflicting with
  $\config{S}{e} \parstepsto \config{S'}{e'}$ and $\Ustore(S') \neq \topS$
  and $\Ustore$ is freeze-safe with $\config{S}{e} \parstepsto
  \config{S'}{e'}$.

  We are required to show that $\config{\Ustore(S)}{e} \parstepsto
  \config{\Ustore(S')}{e'}$.

  The proof is by cases on the rule of the reduction semantics by
  which $\config{S}{e}$ steps to $\config{S'}{e'}$.

  Since $\config{S'}{e'} \neq \error$, we do not need to consider the
  {\sc E-Put-Err} rule.

  The assumption that $\Ustore$ is freeze-safe with $\config{S}{e}
  \parstepsto \config{S'}{e'}$ is only needed in the {\sc
    E-Freeze-Final} and {\sc E-Freeze-Simple} cases.

  \begin{itemize}

  \item Case {\sc E-Beta}:

    Given: $\config{S}{\app{(\lam{x}{e})}{v}} \parstepsto
    \config{S}{\subst{e}{x}{v}}$.

    To show: $\config{\Ustore(S)}{\app{(\lam{x}{e})}{v}} \parstepsto
    \config{\Ustore(S)}{\subst{e}{x}{v}}$.

    Immediate by {\sc E-Beta}.

  \item Case {\sc E-New}:

    Given: $\config{S}{\NEW} \parstepsto
    \config{\extS{S}{l}{\bot}{\frozenfalse}}{l}$.

    To show: $\config{\Ustore(S)}{\NEW} \parstepsto
    \config{\Ustore(\extS{S}{l}{\bot}{\frozenfalse})}{l}$.

    By {\sc E-New}, we have that $\config{\Ustore(S)}{\NEW} \parstepsto
    \config{\extS{(\Ustore(S))}{l'}{\bot}{\frozenfalse}}{l'}$,

    where $l' \notin \dom{\Ustore(S)}$.

    By assumption, $\Ustore$ is non-conflicting with $\config{S}{\NEW}
    \parstepsto \config{\extS{S}{l}{\bot}{\frozenfalse}}{l}$.

    Therefore $l \notin \dom{\Ustore(S)}$.

    Therefore, in
    $\config{\extS{(\Ustore(S))}{l'}{\bot}{\frozenfalse}}{l'}$, we can
    $\alpha$-rename $l'$ to $l$.

    Therefore $\config{\Ustore(S)}{\NEW} \parstepsto
    \config{\extS{(\Ustore(S))}{l}{\bot}{\frozenfalse}}{l}$.

    Also, since $\Ustore$ is non-conflicting with $\config{S}{\NEW}
    \parstepsto \config{\extS{S}{l}{\bot}{\frozenfalse}}{l}$,

    we have that $(\Ustore(\extS{S}{l}{\bot}{\frozenfalse}))(l) =
    (\extS{S}{l}{\bot}{\frozenfalse})(l) =
    \state{\bot}{\frozenfalse}$.

    Hence $\extS{(\Ustore(S))}{l}{\bot}{\frozenfalse} =
    \Ustore(\extS{S}{l}{\bot}{\frozenfalse})$.

    Therefore $\config{\Ustore(S)}{\NEW} \parstepsto
    \config{\Ustore(\extS{S}{l}{\bot}{\frozenfalse})}{l}$, as we were
    required to show.

  \item Case {\sc E-Put}:

    Given: $\config{S}{\putiexp{l}} \parstepsto
    \config{\extSRaw{S}{l}{u_{p_i}(p_1)}}{\unit}$.

    To show: $\config{\Ustore(S)}{\putiexp{l}} \parstepsto
    \config{\Ustore(\extSRaw{S}{l}{u_{p_i}(p_1)})}{\unit}$.

    From the premises of {\sc E-Put}, $S(l) = p_1$.

    Hence $(\Ustore(S))(l) = p'_1$, where $p_1 \leqp p'_1$.

    Next, we want to show that $u_{p_i}(p'_1) \neq \topp$.

    Assume for the sake of a contradiction that $u_{p_i}(p'_1) =
    \topp$.

    Then $u_{p_i}((\Ustore(S))(l)) = \topp$.

    Let $u_{p_j}$ be the state update operation in $\Ustore$ that affects
    the contents of $l$.

    Hence $(\Ustore(S))(l) = u_{p_j}(p_1)$. Then $u_{p_i}(u_{p_j}(p_1)) =
    \topp$.

    Since state update operations commute, $u_{p_j}(u_{p_i}(p_1)) =
    \topp$.
    
    But then $\Ustore(\extSRaw{S}{l}{u_{p_i}(p_1)}) = \topS$,

    which contradicts the assumption that
    $\Ustore(\extSRaw{S}{l}{u_{p_i}(p_1)}) \neq \topS$.

    Hence, $u_{p_i}(p'_1) \neq \topp$.

    Therefore, by {\sc E-Put}, $\config{\Ustore(S)}{\putiexp{l}}
    \parstepsto \config{\extSRaw{(\Ustore(S))}{l}{u_{p_i}(p'_1)}}{\unit}$.

    Since $p'_1 = u_{p_j}(p_1)$, we have that
    $\extSRaw{(\Ustore(S))}{l}{u_{p_i}(p'_1)} =
    \extSRaw{(\Ustore(S))}{l}{u_{p_i}(u_{p_j}(p_1))}$,

    which, since $u_{p_i}$ and $u_{p_j}$ commute, is equal to
    $\extSRaw{(\Ustore(S))}{l}{u_{p_j}(u_{p_i}(p_1))}$.

    Finally, since $u_{p_j}$ is the update operation in $\Ustore$ that
    affects the contents of $l$,

    we have that $\extSRaw{(\Ustore(S))}{l}{u_{p_j}(u_{p_i}(p_1))} =
    \Ustore(\extSRaw{S}{l}{u_{p_i}(p_1)})$, and so the case is satisfied.

  \item Case {\sc E-Get}:

    Given: $\config{S}{\getexp{l}{P}} \parstepsto \config{S}{p_2}$.

    To show: $\config{\Ustore(S)}{\getexp{l}{P}} \parstepsto
    \config{\Ustore(S)}{p_2}$.

    From the premises of {\sc E-Get}, $S(l) = p_1$ and $\incomp{P}$
    and $p_2 \in P$ and $p_2 \leqp p_1$.

    By assumption, $\Ustore(S) \neq \topS$.

    Hence $(\Ustore(S))(l) = p'_1$, where $p_1 \leqp p'_1$.

    By the transitivity of $\leqp$, $p_2 \leqp p'_1$.

    Hence, $(\Ustore(S))(l) = p'_1$ and $\incomp{P}$ and $p_2 \in P$ and
    $p_2 \leqp p'_1$.

    Therefore, by {\sc E-Get},

    $\config{\Ustore(S)}{\getexp{l}{P}} \parstepsto \config{\Ustore(S)}{p_2}$,

    as we were required to show.

  \item Case {\sc E-Freeze-Init}:

    Given: $\config{S}{\freezeafter{l}{Q}{\lam{x}{e}}} \parstepsto
    \\ \config{S}{\freezeafterfull{l}{Q}{\lam{x}{e}}{\setof{}}{\setof{}}}$.

    To show: $\config{\Ustore(S)}{\freezeafter{l}{Q}{\lam{x}{e}}}
    \parstepsto
    \\ \config{\Ustore(S)}{\freezeafterfull{l}{Q}{\lam{x}{e}}{\setof{}}{\setof{}}}$.

    Immediate by {\sc E-Freeze-Init}.

  \item Case {\sc E-Spawn-Handler}:

    Given:

    $\config{S}{\freezeafterfull{l}{Q}{\lam{x}{e_0}}{\setof{e,
          \dots}}{H}} \parstepsto
    \\ \config{S}{\freezeafterfull{l}{Q}{\lam{x}{e_0}}{\setof{\subst{e_0}{x}{d_2},
          e, \dots}} {\{d_2\}\cup H}}$.

    To show:

    $\config{\Ustore(S)}{\freezeafterfull{l}{Q}{\lam{x}{e_0}}{\setof{e,
          \dots}}{H}} \parstepsto
    \\ \config{\Ustore(S)}{\freezeafterfull{l}{Q}{\lam{x}{e_0}}{\setof{\subst{e_0}{x}{d_2},
          e, \dots}} {\{d_2\}\cup H}}$.

    From the premises of {\sc E-Spawn-Handler}, $S(l) =
    \state{d_1}{\status_1}$ and $d_2 \userleq d_1$ and $d_2 \notin H$
    and $d_2 \in Q$.

    By assumption, $\Ustore(S) \neq \topS$.

    Hence $(\Ustore(S))(l) = \state{d'_1}{\status'_1}$ where
    $\state{d_1}{\status_1} \leqp \state{d'_1}{\status'_1}$.

    By Definition~\ref{def:lattice-with-status-bits}, $d_1 \userleq
    d'_1$.

    By the transitivity of $\userleq$, $d_2 \userleq d'_1$.

    Hence $(\Ustore(S))(l) = \state{d'_1}{\status'_1}$ and $d_2 \userleq
    d'_1$ and $d_2 \notin H$ and $d_2 \in Q$.

    Therefore, by {\sc E-Spawn-Handler},

    $\config{\Ustore(S)}{\freezeafterfull{l}{Q}{\lam{x}{e_0}}{\setof{e,
          \dots}}{H}} \parstepsto
    \\ \config{\Ustore(S)}{\freezeafterfull{l}{Q}{\lam{x}{e_0}}{\setof{\subst{e_0}{x}{d_2},
          e, \dots}} {\{d_2\}\cup H}}$,

    as we were required to show.

  \item Case {\sc E-Freeze-Final}:

    Given: $\config{S}{\freezeafterfull{l}{Q}{\lam{x}{e_0}}{\setof{v,
          \dots}}{H}} \parstepsto
    \\ \config{\extS{S}{l}{d_1}{\frozentrue}}{d_1}$.

    To show:
    $\config{\Ustore(S)}{\freezeafterfull{l}{Q}{\lam{x}{e_0}}{\setof{v,
          \dots}}{H}} \parstepsto
    \\ \config{\Ustore(\extS{S}{l}{d_1}{\frozentrue})}{d_1}$.

    From the premises of {\sc E-Freeze-Final}, $S(l) =
    \state{d_1}{\status_1}$.

    We have two cases to consider:
    \begin{itemize}
    \item $\status_1 = \frozentrue$:

      In this case, $S(l) = \state{d_1}{\frozentrue}$.

      Let $u_{p_i}$ be the state update operation in $\Ustore$ that
      affects the contents of $l$.

      Hence $(\Ustore(S))(l) = u_{p_i}(\state{d_1}{\frozentrue})$.

      We know from Definition~\ref{def:set-of-state-update-operations}
      that $u_{p_i}(\state{d_1}{\frozentrue})$ is either
      $\state{d_1}{\frozentrue}$ or $\state{\top}{\frozenfalse}$.

      But if $u_{p_i}(\state{d_1}{\frozentrue}) =
      \state{\top}{\frozenfalse}$, then
      $\Ustore(\extS{S}{l}{d_1}{\frozentrue}) = \topS$, which contradicts
      our assumption that $\Ustore(\extS{S}{l}{d_1}{\frozentrue}) \neq
      \topS$.

      Hence $u_{p_i}(\state{d_1}{\frozentrue}) =
      \state{d_1}{\frozentrue}$.

      Hence $(\Ustore(S))(l) = \state{d_1}{\frozentrue}$, and we already
      have from the premises of {\sc E-Freeze-Final} that
      $\forall{d_2} ~.~ ( {d_2 \userleq d_1 \land d_2 \in Q}
      \Rightarrow d_2 \in H)$.

      Hence, by {\sc E-Freeze-Final}, we have that

      $\config{\Ustore(S)}{\freezeafterfull{l}{Q}{\lam{x}{e_0}}{\setof{v,
            \dots}}{H}} \parstepsto
      \config{\extS{(\Ustore(S))}{l}{d_1}{\frozentrue}}{d_1}$.

      Finally, since $u_{p_i}$ is the state update operation in $\Ustore$
      that affects the contents of $l$,

      and $u_{p_i}(\state{d_1}{\frozentrue}) =
      \state{d_1}{\frozentrue}$, we have that
      $\extS{(\Ustore(S))}{l}{d_1}{\frozentrue}$ is equal to
      $\Ustore(\extS{S}{l}{d_1}{\frozentrue})$, and so the case is
      satisfied.

    \item $\status_1 = \frozenfalse$:

      By assumption, $\Ustore$ is freeze-safe with
      $\config{S}{\freezeafterfull{l}{Q}{\lam{x}{e_0}}{\setof{v,
            \dots}}{H}} \parstepsto
      \config{\extS{S}{l}{d_1}{\frozentrue}}{d_1}$.

      Therefore $\Ustore$ acts as the identity on the contents of any
      locations that change status during the transition.

      Since $\status_1 = \frozenfalse$, the contents of $l$ change
      status during the transition.

      Therefore $\Ustore$ acts as the identity on the contents of $l$.

      Hence $(\Ustore(S))(l) = S(l) = \state{d_1}{\status_1}$, and we
      already have from the premises of {\sc E-Freeze-Final} that
      $\forall{d_2} ~.~ ( {d_2 \userleq d_1 \land d_2 \in Q}
      \Rightarrow d_2 \in H)$.

      Hence, by {\sc E-Freeze-Final}, we have that

      $\config{\Ustore(S)}{\freezeafterfull{l}{Q}{\lam{x}{e_0}}{\setof{v,
            \dots}}{H}} \parstepsto
      \config{\extS{(\Ustore(S))}{l}{d_1}{\frozentrue}}{d_1}$.

      Finally, since $\Ustore$ acts as the identity on the contents of
      $l$, we have that $\extS{(\Ustore(S))}{l}{d_1}{\frozentrue}$ is
      equal to $\Ustore(\extS{S}{l}{d_1}{\frozentrue})$, and so the case
      is satisfied.
    \end{itemize}

  \item Case {\sc E-Freeze-Simple}:

    Given: $\config{S}{\freeze{l}} \parstepsto
    \config{\extS{S}{l}{d_1}{\frozentrue}}{d_1}$.

    To show: $\config{\Ustore(S)}{\freeze{l}} \parstepsto
    \config{\Ustore(\extS{S}{l}{d_1}{\frozentrue})}{d_1}$.

    From the premises of {\sc E-Freeze-Simple}, $S(l) =
    \state{d_1}{\status_1}$.

    We have two cases to consider:
    \begin{itemize}
    \item $\status_1 = \frozentrue$:

      In this case, $S(l) = \state{d_1}{\frozentrue}$.

      Let $u_{p_i}$ be the state update operation in $\Ustore$ that
      affects the contents of $l$.

      Hence $(\Ustore(S))(l) = u_{p_i}(\state{d_1}{\frozentrue})$.

      We know from Definition~\ref{def:set-of-state-update-operations}
      that $u_{p_i}(\state{d_1}{\frozentrue})$ is either
      $\state{d_1}{\frozentrue}$ or $\state{\top}{\frozenfalse}$.

      But if $u_{p_i}(\state{d_1}{\frozentrue}) =
      \state{\top}{\frozenfalse}$, then
      $\Ustore(\extS{S}{l}{d_1}{\frozentrue}) = \topS$, which contradicts
      our assumption that $\Ustore(\extS{S}{l}{d_1}{\frozentrue}) \neq
      \topS$.

      Hence $u_{p_i}(\state{d_1}{\frozentrue}) =
      \state{d_1}{\frozentrue}$.

      Hence $(\Ustore(S))(l) = \state{d_1}{\frozentrue}$.

      Hence, by {\sc E-Freeze-Simple}, we have that
      $\config{\Ustore(S)}{\freeze{l}} \parstepsto
      \\ \config{\extS{(\Ustore(S))}{l}{d_1}{\frozentrue}}{d_1}$.

      Finally, since $u_{p_i}$ is the state update operation in $\Ustore$
      that affects the contents of $l$,

      and $u_{p_i}(\state{d_1}{\frozentrue}) =
      \state{d_1}{\frozentrue}$, we have that
      $\extS{(\Ustore(S))}{l}{d_1}{\frozentrue}$ is equal to
      $\Ustore(\extS{S}{l}{d_1}{\frozentrue})$, and so the case is
      satisfied.

    \item $\status_1 = \frozenfalse$:

      By assumption, $\Ustore$ is freeze-safe with $\config{S}{\freeze{l}}
      \parstepsto \config{\extS{S}{l}{d_1}{\frozentrue}}{d_1}$.

      Therefore $\Ustore$ acts as the identity on the contents of any
      locations that change status during the transition.

      Since $\status_1 = \frozenfalse$, the contents of $l$ change
      status during the transition.

      Therefore $\Ustore$ acts as the identity on the contents of $l$.

      Hence $(\Ustore(S))(l) = S(l) = \state{d_1}{\status_1}$.

      Hence, by {\sc E-Freeze-Simple}, we have that
      $\config{\Ustore(S)}{\freeze{l}} \parstepsto
      \\ \config{\extS{(\Ustore(S))}{l}{d_1}{\frozentrue}}{d_1}$.

      Finally, since $\Ustore$ acts as the identity on the contents of
      $l$, we have that $\extS{(\Ustore(S))}{l}{d_1}{\frozentrue}$ is
      equal to $\Ustore(\extS{S}{l}{d_1}{\frozentrue})$, and so the case
      is satisfied.
    \end{itemize}

  \end{itemize}
\end{proof}
