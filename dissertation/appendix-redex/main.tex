\chapter{PLT Redex Models of $\lambdaLVar$ and $\lambdaLVish$}\label{app:plt-redex}

\TODO{Edit this text.}

We have developed a runnable version of the LVish
calculus\footnote{Available at
  \url{http://github.com/iu-parfunc/lvars}.} using the PLT Redex
semantics engineering toolkit \cite{redex-book}.  In the Redex of
today, it is not possible to directly parameterize a language
definition by a lattice.\footnote{See discussion at
  \url{http://lists.racket-lang.org/users/archive/2013-April/057075.html}.}
Instead, taking advantage of Racket's syntactic abstraction
capabilities, we define a Racket macro,
\texttt{define-LVish-language}, that wraps a template implementing the
lattice-agnostic semantics of $\lambdaLVish$, and
takes the following arguments:
\begin{itemize}
\item a \emph{name}, which becomes the \emph{lang-name} passed to
  Redex's \texttt{define-language} form;
\item a \emph{``downset'' operation}, a Racket-level procedure that
  takes a lattice element and returns the (finite) set of all lattice
  elements that are below that element (this operation is used to
  implement the semantics of $\FAW$, in particular, to determine when
  the {\sc E-Freeze-Final} rule can fire);
\item a \emph{lub operation}, a Racket-level procedure that takes two
  lattice elements and returns a lattice element; and
\item a (possibly infinite) set of \emph{lattice elements} represented
  as Redex \emph{patterns}.
\end{itemize}
Given these arguments, \texttt{define-LVish-language} generates a
Redex model specialized to the application-specific lattice in
question. For instance, to instantiate a model called \texttt{nat},
where the application-specific lattice is the natural numbers with
\texttt{max} as the least upper bound, one writes:
\[
\texttt{(define-LVish-language nat downset-op max natural)}
\]
where \texttt{downset-op} is separately defined.  Here,
\texttt{downset-op} and \texttt{max} are Racket procedures.
\texttt{natural} is a Redex pattern that has no meaning to Racket
proper, but because \texttt{define-LVish-language} is a macro,
\texttt{natural} is not evaluated until it is in the context of
Redex.\lk{This might be too much information, or a little confusing.
  It's a nice illustration of the power of macros, though.  I'd
  welcome suggestions for how to word it differently.}

\TODO{Freshen up the Redex models of $\lambdaLVar$ and $\lambdaLVish$
  and add them here.}




