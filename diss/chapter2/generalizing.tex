\section{Generalizing the \lstinline|put| and \lstinline|get| operations} \label{s:lvars-generalizing}

\TODO{Write this section!}

\subsection{Generalizing from least-upper-bound writes to inflationary, commutative writes}

\TODO{Write this subsection!}

\lk{This will be about generalized inflationary+commutative writes,
  rather than least-upper-bound writes (lub is a special case of
  this).}

\TODO{Say something about how the lattice in
  Figure~\ref{f:lvars-example-lattices}(c) has different semantics if
  we have least-upper-bound writes or incremental writes, and how
  incremental writes may in fact be what is desired.  Possibly cite
  the CRDTs work and forward-reference Chapter~\ref{ch:distributed}.}

\lk{Here I can probably reuse some material from the PLDI and DISC
  papers and from my thesis proposal.}

\subsection{A more general formulation of threshold sets}

\TODO{Write this subsection!}

\lk{This will be about generalized threshold sets, rather than simple
  threshold sets (i.e., from our DISC '14 submission); explain how the
  threshold sets that we've seen so far are a special case of this).}

\subsection{Generalizing from threshold sets to threshold functions}


\TODO{Write this subsection!}

\lk{This will be about "threshold functions", rather than either
  simple threshold sets or generalized threshold sets (i.e., partial
  functions that take a lattice element and are undefined for all
  inputs that are not at or above a given point in the lattice, and
  constant for all inputs that are at or above that point; both kinds
  of threshold sets are a special case of this, afaict).}
