\section{Determinism of threshold queries}\label{s:distributed-results}

Neither eventual consistency nor strong eventual consistency imply
that \emph{intermediate} results of the same query $q$ on different
replicas of a threshold CvRDT will be deterministic.  For
deterministic intermediate results, we must use the threshold query
method $t$.  We can show that $t$ is deterministic \emph{without}
requiring that the same updates have been delivered at the replicas in
question at the time that $t$ runs.

Theorem~\ref{thm:determinism-of-threshold-queries} establishes a
determinism property for threshold queries of CvRDTs, porting the
determinism result previously established for threshold reads for
LVars to a distributed setting.

\begin{thm}[Determinism of Threshold Queries]
  \label{thm:determinism-of-threshold-queries}
  Suppose a given threshold query $t$ on a given threshold CvRDT
  returns a set of activation states $S_a$ when executed at a replica
  $i$.  Then, assuming eventual delivery and that no replica's state
  is ever $\top$ at any point in the execution:
  \begin{enumerate}
  \item \label{thm:this-replica} $t$ will always return $S_a$ on
    subsequent executions at $i$, and
  \item \label{thm:any-replica} $t$ will \emph{eventually} return
    $S_a$ when executed at \emph{any} replica, and will \emph{block}
    until it does so.
  \end{enumerate}
\end{thm}
\begin{proof}
The proof relies on transitivity of $\leq$ and eventual delivery of
updates; see
Section~\ref{section:determinism-of-threshold-queries-proof} for the
complete proof.
\end{proof}

Although Theorem~\ref{thm:determinism-of-threshold-queries} must
assume eventual delivery, it does \emph{not} need to assume strong
convergence or even ordinary convergence.  It so happens that we have
strong convergence as part of strong eventual consistency of threshold
CvRDTs (by
Theorem~\ref{thm:strong-eventual-consistency-of-threshold-cvrdts}),
but we do not need it to prove
Theorem~\ref{thm:determinism-of-threshold-queries}.  In particular,
there is no need for replicas to have the same state in order to
return the same result from a particular threshold query.  The
replicas merely both need to be above an activation state from a
unique set of activation states in the query's threshold set.  Indeed,
the replicas' states may in fact trigger \emph{different} activation
states from the same set of activation states.

Theorem~\ref{thm:determinism-of-threshold-queries}'s requirement that
no replica's state is ever $\top$ rules out situations in which
replicas disagree in a way that cannot be resolved normally.  Recall
from Section~\ref{subsection:lvars-errors-and-observable-determinism}
that in the LVars model, when a program contains conflicting writes
that would cause an LVar to reach its $\top$ state, a threshold read
in that program \emph{can} behave nondeterministically.  However,
since in our definition of observable determinism, only the final
outcome of a program counts, this nondeterministic behavior of @get@
in the presence of conflicting writes is not observable: such a
program would always have $\error$ as its final outcome.  In our
setting of CvRDTs, though, we do not have a notion of ``program'', nor
of the final outcome thereof.  Rather than having to define those
things and then define a notion of observable determinism based on
them, I rule out this situation by assuming that no replica's state
goes to $\top$.



