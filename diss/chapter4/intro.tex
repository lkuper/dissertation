We want the programming model of Chapters~\ref{ch:lvars}
and~\ref{ch:quasi} to be realizable in practice.  If the determinism
guarantee offered by LVars is to do us any good, however, we need to
add LVars to a parallel programming environment that is already known
to be deterministic.

The \emph{monad-par} Haskell library~\cite{monad-par}, which provides
the @Par@ monad, is one such known-deterministic parallel programming
environment.  The existing @Par@ monad is an appealing starting point
for a practical implementation of LVars because the inter-task
communication that it allows is through IVars, which, as we have seen,
are a special case of LVars.  Implementing the basic LVars model is
therefore a matter of generalizing the existing @Par@ monad
implementation.

The @Par@ monad approach is also appealing because it is implemented
entirely as a library in Haskell, with a library-level scheduler.
This modularity makes it possible to make changes to the @Par@
scheduling strategy (which we will need to do in order to support
LVars) without having to make any modifications to GHC or its run-time
system.

Finally, Haskell is in general an appealing substrate for implementing
the LVars programming model because it is pure by default, and its
type system enforces separation of pure and effectful code.  For the
determinism guarantee of the LVars model (or the quasi-determinism
guarantee of the extended model) to hold, the only side effects
allowed must be @put@ and @get@ operations on LVars. \lk{Should we say
that both \lstinline|put| and \lstinline|get| are effects, or
just \lstinline|put|?}  Implementing the LVars model as a Haskell
library makes it possible to provide compile-time guarantees about
determinism and quasi-determinism, because programs written using the
library run in the @Par@ monad and can therefore only perform the side
effects that we sanction.\footnote{Haskell is often advertised as a
purely functional programming language, that is, one without side
effects, but it is perhaps more useful to think of it as a language
that gets other effects out of the way so that one can add one's own
effects!}

In this chapter, I describe the \emph{LVish} library, a Haskell
library for practical deterministic and quasi-deterministic parallel
programming with LVars.  We have already had a taste of what it is
like to use the LVish library---Section~\ref{s:quasi-informal} even
gives an example of an LVish Haskell program---but we have not yet had
a systematic introduction to the LVish Haskell library API.  In
Section~\ref{s:lvish-api}, I discuss the high-level design goals of
LVish and describe the LVish API, and I discuss the parallel
breadth-first traversal of Section~\ref{s:quasi-informal} in more
detail.  Next, in Section~\ref{s:lvish-internals}, I describe how the
LVish library itself is implemented.  Finally, in
Sections~\ref{s:lvish-k-cfa} and~\ref{s:lvish-phybin}, I present two
case studies of parallelizing existing Haskell programs by porting
them to LVish.  First, in Section~\ref{s:lvish-k-cfa}, I describe
using LVish to parallelize a control-flow analysis ($k$-CFA)
algorithm.  Second, in Section~\ref{s:lvish-phybin}, I describe using
LVish to parallelize
\emph{PhyBin}, a bioinformatics application that relies heavily on a
(parallelizable) tree-edit distance algorithm.
