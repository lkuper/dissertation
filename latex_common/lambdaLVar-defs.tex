\newcommand{\LVarsDefEqStore}{
\begin{definition}[equality of stores]\label{def:lvars-eqstore}
  Two stores $S$ and $S'$ are \emph{equal} iff:
  \begin{enumerate}
    \item $S = \topS$ and $S' = \topS$, or
    \item $\dom{S} = \dom{S'}$ and for all $l \in \dom{S}$, $S(l) =
      S'(l)$.
  \end{enumerate}
\end{definition}
}

\newcommand{\LVarsDefLeqStore}{
\begin{definition}[store ordering]\label{def:lvars-leqstore}
  A store $S$ is \emph{less than or equal to} a store $S'$ (written
  $\leqstore{S}{S'}$) iff:
  \begin{itemize}
  \item $S' = \topS$, or
  \item $\dom{S} \subseteq \dom{S'}$ and for all $l
    \in \dom{S}$, $S(l) \userleq S'(l)$.
  \end{itemize}
\end{definition}
}

\newcommand{\LVarsDefLeqStoreCnC}{
\begin{definition}[store ordering, Featherweight CnC]\label{def:lvars-leqstore-cnc}
  A store $S$ is \emph{less than or equal to} a store $S'$ (written
  $\leqstore{S}{S'}$) iff $\dom{S} \subseteq \dom{S'}$ and for all $l
  \in \dom{S}$, $S(l) = S'(l)$.
\end{definition}
}

\newcommand{\LVarsDefLubStore}{
\begin{definition}[least upper bound of stores]\label{def:lvars-lubstore}

  The \emph{least upper bound} of two stores $S_1$ and $S_2$ (written
  $\lubstore{S_1}{S_2}$) is defined as follows:
  
  \begin{itemize}
  \item $\lubstore{S_1}{S_2} = \topS$ iff there exists some $l \in
    \dom{S_1} \cap \dom{S_2}$ such that $\userlub{S_1(l)}{S_2(l)} = \top$.
  \item Otherwise, $\lubstore{S_1}{S_2}$ is the store $S$ such that:

  \begin{itemize}
  \item $\dom{S} = \dom{S_1} \cup \dom{S_2}$, and
  \item For all $l \in \dom{S}$:
  \end{itemize}
    \begin{displaymath}
      S(l) = \left\{ \begin{array}{ll}
        \userlub{S_1(l)}{S_2(l)} & \textrm{if $l \in \dom{S_1} \cap \dom{S_2}$} \\
        S_1(l) & \textrm{if $l \notin \dom{S_2}$} \\
        S_2(l) & \textrm{if $l \notin \dom{S_1}$}
        \end{array} \right.
    \end{displaymath}
  \end{itemize}
\end{definition}
}

\newcommand{\LVarsDefPermutation}{
\begin{definition}[permutation]\label{def:lvars-permutation}
  A \emph{permutation} is a function $\pi : \Loc \rightarrow \Loc$
  such that:
  \begin{enumerate}
  \item it is invertible, that is, there is an inverse function
    $\piinv : \Loc \rightarrow \Loc$ with the property that $\pi(l) =
    l'$ iff $\piinv(l') = l$; and
    \item it is the identity on all but finitely many elements of
      $Loc$.
  \end{enumerate}
\end{definition}
}

\newcommand{\LVarsDefPermutationExpression}{
\begin{definition}[permutation (expressions)]\label{def:lvars-permutation-expression}
  A \emph{permutation} of an expression $e$ is a function $\pi$ defined as
  follows:
  \begin{displaymath}
    \begin{array}{ll}
      \pi(x) & \defeq x \\
      \pi(\unit) & \defeq \unit \\
      \pi(d) & \defeq d \\
      \pi(l) & \defeq \pi(l) \\
      \pi(T) & \defeq T \\
      \pi(\lam{x}{e}) & \defeq \lam{x}{\pi(e)} \\
      \pi(\app{e_1}{e_2}) & \defeq \app{\pi(e_1)}{\pi(e_2)} \\
      \pi(\getexp{e_1}{e_2}) & \defeq \getexp{\pi(e_1)}{\pi(e_2)} \\
      \pi(\putexp{e_1}{e_2}) & \defeq \putexp{\pi(e_1)}{\pi(e_2)} \\
      \pi(\NEW) & \defeq \NEW \\
    \end{array}
  \end{displaymath}
  \lk{It's fine to just say that $\pi(\NEW)$ is $\NEW$; we only care
    about renaming it if it has already been allocated!  If it's just
    an unevaluated $\NEW$ expression, then there's nothing to do.}
\end{definition}
}

\newcommand{\LVarsDefPermutationStore}{
\begin{definition}[permutation (stores)]\label{def:lvars-permutation-store}
  A \emph{permutation} of a store $S$ is a function $\pi$ defined as
  follows:
  \begin{displaymath}
    \begin{array}{ll}
      \pi(\topS) & \defeq \topS \\
      \pi(\store{\storebindingRaw{l_1}{d_1}, \dots,
        \storebindingRaw{l_n}{d_n}}) & \defeq
      \store{\storebindingRaw{\pi(l_1)}{d_1}, \dots,
        \storebindingRaw{\pi(l_n)}{d_n}} \\
    \end{array}
  \end{displaymath}
\end{definition}
}

\newcommand{\LVarsDefPermutationConfiguration}{
\begin{definition}[permutation (configurations)]\label{def:lvars-permutation-configuration}
  A \emph{permutation} of a configuration $\config{S}{e}$ is a function $\pi$
  defined as follows: if $\config{S}{e} = \error$, then
  $\pi(\config{S}{e}) = \error$; otherwise, $\pi(\config{S}{e}) =
  \config{\pi(S)}{\pi(e)}$.
\end{definition}
}

\newcommand{\LVarsDefNonConflicting}{
\begin{definition}[non-conflicting store]\label{def:lvars-non-conflicting}
  A store $S''$ is \emph{non-conflicting} with the transition
  $\config{S}{e} \parstepsto \config{S'}{e'}$ iff $(\dom{S'} -
  \dom{S}) \cap \dom{S''} = \emptyset$.
\end{definition}
}
